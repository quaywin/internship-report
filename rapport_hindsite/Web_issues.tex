Web issues.

 
To understand the issues that arises when developing a web site or web application, let me first remind you how a user can access a web page. 
- First, the client (the user's web browser) requests a web page located on a web server. The function of this web server is to deliver the requested web page to the client. This means delivery of an HTML document along with additional content, 
such as images, stylesheets and JavaScripts. 
- The client's web browser application then processes the content and displays it for the user. 
While the server will always deliver the same content for the same requested page, two different users might see two different things on their computer. 
The reason is that there are more than one web browser software application existing, and while they try to make it look the same, the process behind the display and interpretation of the content is not. Set aside Html content, which is quite processed in the same way among every browsers, issues arises with interpretation of
JavaScript and stylsheets (Css). For example, Firefox will understand the css property moz-border-radius while Google Chrome and Internet Explorer will not. Also, Firefox will not understand the JavaScript Object window.event while Google Chrome and Internet Explorer will. There are dozens of example like these ones.
While you can develop the server side part of the application in one language and not worry about its ability to always deliver the same content, you can't afford to develop the client side application for only one browser. This would result in many internet users not to be able to see the content as intended. (cf fig on browser statistics).
% w3c table showing market share of web browser on the site of the page %
And as a web development company, that is not an option. (your clients won't like that if you tell them that only 30 
% of the internet users will be able to use their website.)
Web application that behaves the same way in multiple browser are called "cross-browser". 
My job was of course to make sure that every web content I created was cross-browser (at least for IE7, IE8, Chrome and FF).
At first, this can look like a hard task to accomplish, because you would need to be aware of each browser specificities. Fortunatly, some tools exists to help you reach that goal.

1/ JavaScript

JavaScript has some big differences between browsers. For example, to attach an event to an element, Firefox and Chrome uses the function addEventListener. Internet Explorer will not understand this function as it uses attachEvent for the same purpose. (cf figure with some differences on styling an element in javascript http://www.quirksmode.org/dom/w3c_css.html).
Also, one of the purpose of JavaScript is being able to handle all the html elements in the web page to increase the user experience on the website. And again, there are some slight differences between every browser.
Because of all those disparities, people started to build APIs, which would considerably reduce the knowledge needed of those disparities to run a cross-browser JavaScript code. On of the most famous is jQuery. Simple, yet powerful, you can find in jQuery all the Javascript functionalities, encapsulated in cross-browser functions. Ex: instead of having to test if the browser is IE to call the function attachEvent instead of addEventListener, you can just call the jQuery method bind to attach an event to an element. No need to learn the supported functions of each browser, you just need to learn a simple Api. jQuery even simplifies some JavaScript functionalities, such as the use of Ajax. (cf piece of code w w/o jQuery).

2/ Css
Unlike in JavaScript, there is no API to help you with css (mainly because Css is not a programming language). The huge issues in Css arises when you want to make it work on Internet explorer. Usually, you are gonna develop with Chrome or Firefox (Internet Explorer can be very slow and its a pain in the ass to develop on) and the same css is going to work on each other. It might even work in Internet Explorer 8 (with some really little adjustment on precise case). But most of the time, you are going to have to do some adjustments so that it works on Internet Explorer 7. Those adjustment might be to just rethink the way of styling an element to completely tweaking the css just for Internet Explorer. The last is called "css hacking". And fortunatly, probably because Microsoft knows that his css handling is not that good, it introduced what is called conditional comments. It looks like a simple html comments in the html content, but IE knows it is aimed at him and can process it to add a specific stylesheet. Other browsers will just ignore it, as it is a meer comment for them. (cg example html + speakk about prefix * _ etc).