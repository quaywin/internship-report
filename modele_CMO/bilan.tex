\chapter{Conclusion}

Ce chapitre est la conclusion de tout le travail réalisé dans l'équipe R\&D de Smartesting.
C'est aussi, la conclusion d'un challenge personnel de reprise d'étude après cinq années passées dans la vie active.
Et pour finir, c'est l'occasion de dire, la chance que j'ai d'entrer dans l'équipe Smartesting.

\section{Professionnelle}

Le stage s'est déroulé dans de très bonnes conditions.
J'ai appris à travailler d'une façon très différente.
Et cela m'a très fortement intéressé tout au long de celui-ci.

\subsection{Objectifs}

Les missions confiées durant le stage ont toutes été remplies.
La modification du \build a permis de fragmenter l'application en plugins réutilisables.
Cela apporte aussi plus de possibilités pour le déploiement de l'application.

\subparagraph*{}
La création d'un plugin d'exportation de modèle pour Papyrus a été réalisée avec succès.
Il est désormais possible d'exporter un modèle Papyrus vers Test Designer. 
Quant à la réorganisation des modules en plugins et l'utilisation des points d'extensions, cela devrait permettre le développement plus rapide et plus simple de fonctionnalités spécifiques à chaque modeleur.

\subsection{Échange de connaissance}

J'ai reçu de la part des collègues de Smartesting bien plus que des conseils ou une méthode de programmation particulière, mais une culture de programmation commune.
Par exemple la programmation par intention : il s'agit d'écrire ce que fait quelque chose en langage de programmation Java sans avoir besoin d'insérer de commentaires dans tout le code.
L'intérêt est de ne pas avoir à maintenir la mise à jour des commentaires lorsque l'on procède à un redécoupage fonctionnel.

\subparagraph{}
La transmission de connaissances n'a pas marché que dans un sens.
J'ai en effet partagé mes connaissances de la plateforme Eclipse avec l'équipe.
J'ai même réussi le challenge d'améliorer le temps de test des plugins pour les modeleurs.

Initialement, la technique employée consistait à installer de manière conventionnelle le plugin d'exportation pour le tester.
En proposant, une technique différente le temps de test est passé d'un maximum de quinze minutes à six minutes.
C'est en fait le temps personnel que j'ai passé à étudier la plateforme Eclipse qui m'a permis de réaliser tout ça.

\subsection{L'agilité}
Avant d'arriver chez Smartesting et même avant le Master, j'avais travaillé cinq ans en entreprise.
La méthode de développement était très classique et pesante.
Je pense aujourd'hui que le développement aurait pu être plus ``fun'' avec la pratique de l'agilité et tout aussi productif.
Chez Smartesting, je suis heureux d'avoir pu participer à cette expérience.

\subparagraph*{}
J'ai énormément appris des méthodes ``Agile'', je pense être capable de proposer cette méthode de travail auprès de mes prochains collègues de travail.
J'ai trouvé chez Smartesting une très bonne cohésion de groupe qui, je pense, est due aux discussions et aux conditions de travail dans la bonne humeur.

\section{Personnelle}

\subsection{Enjeu de carrière}

Ce stage fut pour moi, un enjeu de carrière.
Je suis arrivé au terme des trois années de reprise d'étude que je m'étais fixée avec raison.
Grâce à l'Université de Franche-Comté et de la formation continue, j'ai atteint le niveau de compétence que je m'étais juré d'obtenir.

\subparagraph*{}
L'enseignement que j'ai reçu m'a totalement convaincu.
Je pense que c'est grâce à mon bagage en entreprise, que chaque difficulté me semblait nécessaire et juste.
De tout mon cursus, c'est bizarrement les matières sur les tests fonctionnels (le B) qui m'ont toujours posées le plus de difficulté.
Et pourtant, c'est le domaine dans lequel j'ai travaillé.

\subsection{Le futur}
Pour finir, je retire une grande satisfaction personnelle d'avoir accompli autant de choses aussi intéressantes durant une période de stage aussi courte.
Et je suis content que Smartesting puisse me garder au sein de l'équipe R\&D pour une durée de sept mois.

\subparagraph*{}
Dorénavant, je suis certain d'avoir les compétences requises pour n'importe quel poste en informatique.
Suffit de s'en donner les moyens.
