\chapter{Activités confiées pendant le stage}
Pendant mon stage j'ai participé à beaucoup de fiches différentes, il serait long et ennuyeux de les détailler toutes sachant que je participai à entre 1 et 4 fiches par semaine. Je présenterai dans les grandes lignes les fonctionnalités majeure auxquelles j'ai participé.

\subsection{Développement sur le code de production}
Parmi les activité qui m'ont été confiées, j'ai participé à la réalisation de fiches blanches (cf. \ref{agile:fiches}). Le code produit est appelé ``code de production'' car il couvre les besoins fonctionnels exprimés par le client XP. La première étape dans le travail sur une fiche est de préciser le périmètre fonctionnel avec le client XP. Ceci permet aux développeurs de connaitre son besoin précis. Ensuite on crée les tests unitaires qui permettront de savoir si la fonctionalité est opérationnelle. Les membres du binôme peuvent prendre le clavier au moment où ils le souhaitent et 
Au début de mon stage, la première fiche à laquelle j'ai participé traitait d'un bug d'IHM\footnote{Interface homme-machine}.TODO : Correction bug affichage no changes, travail sur Setup/teardown, 
\subsection{Documentation}

\subsection{Validation}
La validation des fiches est obligatoire pour considérer que la fonctionnalité lui correspondant est terminée. Elle consiste au test de la fonctionnalité sur l'application après commit. La validation se fait généralement en binôme et il faut chercher toutes les combinaisons en rapport avec la nouvelle fonctionnalité qui sont susceptible d'échouer, de vérifier si toutes les objectifs de la fiche sont remplis. Après avoir testé un maximum de cas de figures il faut faire appel au Client XP qui va s'assurer que l'engagement a été rempli. Une fois cela fait, la fiche est terminée et son nombre de points de vélocité peut être ajouté à ce qui a déjà été fait
\subsection{Amélioration du code existant (refactoring)}
TODO : utilité ,exemple, 
\subsection{Administration système}
Sur une courte période j'ai effectué des t\^aches d'administration système. En particulier au moment de l'intégration des Google Apps dans le fonctionnement de Smartesting. La necessité de partager des calendriers et de pouvoir y accéder via des plateformes mobiles a amené Smartesting à envisager d'utiliser Google Apps. Ainsi, en binôme avec Olivier, nous avons appréhendé le panneau d'administration ainsi que les différents services utilisables. Chaque calendrier donne la possibilité d'être exporté, ainsi nous avons pu réaliser une routine de backup\footnote{sauvegarde automatique}.
\subsection{Meeting corporate}
TODO : definition, utilité, ma participation ...
\subsection{Visite de BNP Paribas}
Mes impressions, les decisions, Smart et BNP ...

\subsection{Amélioration du process, évolution du fonctionnement de l'équipe}
J'ai été ammené à l'occasion de retrospectives ou de réunions à réfléchir sur le processus de développement au sein de l'équipe. L'équipe de R\&D est très concernée par l'amélioration du processus et toute pratique peut être remise en cause si elle ne convient pas a l'équipe. Chaque décision quelle qu'elle soit doit être approuvée par toute l'équipe avant d'être prise. Je parlerai plus en détail à travers d'exemple du type de décisions qui sont prises sur ce sujet.

Tableau des évolution des pratiques XP
\begin{table}[!ht]
	\caption{\label{tableau:evolPratXP}Evolution des pratiques XP au cours du stage}
	\begin{tabular}{|l|c|c|}
		\hline
		Pratique & Début du stage & Fin du stage\\
		\hline
		Pair programming & \tick & \tick \\
		Itération & 2 semaines & 1 semaine \\
		Intégration continue & \tick & \tick \\
		Niko Niko & \tick & \badtick \\
		Lecture & \tick & \badtick \\
		Point perso & hebdomadaire & nouvelles experimentations\\
		Pomodoro & \badtick & \tick \\
		Test driven development & \tick & \tick \\
		``Done done'' & théorique & adopté et normalisé\\
		``No Bugs'' & imprécis & engagement \\
		Slack Time & \badtick & adopté et normalisé\\
		Rétrospective & 1 à 2h le lundi matin & Timeboxée et juste après la livraison\\
		Point technique & assez réguliers & moins nombreux\\
		Veilleur & \tick & Robin cumule son rôle\\
		Batman/Robin & \badtick & \tick \\
		\hline
	\end{tabular}
\end{table}


TODO : Pomodoro, Decisions lors de retrospectives, Iteration 1 semaine, Slack, reduction de vélocité, Done-done, rédaction(et simplification) des standards, Objectifs R\& D, ...