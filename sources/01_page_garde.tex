%\documentclass[a4paper]{classe_de_document} pour le format a4
%texclipse pour eclipse
\documentclass{report}
%classes dispos : book, report, article, slides, letter
\usepackage{T1}{fontenc}
%permet de gerer directement els accents
\usepackage{frenchb}{babel}
%regles de typographie française
%liste des caractères clé : '\'début de macro, '%' commentaire , '~' espace insécable, '{' début de groupe, '}' fin de groupe, '$' mode mathématique, '_' indice, '^' exposant, '#' paramètre de macro, '&' colonne de tableau
\setlength{\parindent}{3cm} % indentation des paragraphes
\setlength{\parskip}{18pt plus4pt minus2pt} % dimension des sauts entre les paragraphes
\begin{document}
Exemple1 :
Mes premiers pas avec \LaTeX{} sont un peu émouvants.
%espace insécable si on veut séparer la macro du reste :
Mes premiers pas avec \LaTeX{}~sont un peu émouvants.
% plein d'espaces sont réduits à un seul
Second exemple {\LaTeX} pour separer macro du texte.
Troisième exemple pour separer \LaTeX\ du reste.
%espace insécable pour avoir la bonne coupure.
Troisième exemple pour separer~\LaTeX\ du reste.
% obtention des caractères réservés
% '\' = \(\backslash\)
% '$' = \$
% '%' = \%
% '_' = \_
% '~' = \~{}
% '^' = \^{}
% '{' = \{
% '}' = \}
% '#' = \#
% '&' = \&
% accentuer les lettres \^lettre, ex : \^e
% rappel les capitals doivent être accentuées
% accent aigu = \'
% trema = \"
% grave \`
% circonflexe = \^
% Macros de base :
% \oe \OE \o \O \S \P \copyright
% groupes pour les fonts : tiny, scriptsize, footnotesize, small, normalsize, large, Large, LARGE, huge, Huge
{\tiny groupe de taille tiny}
{\normalsize groupe de taille normale}
{\large groupe de taille large}
%types de fonts :
Macro I  Macro II            Résultat
           Macros pour la forme
\textup \upshape foerme droit
\textit \itshape forme italique
\textsl \slshape penché
\textsc \scshape petite capitale
          Macros pour la graisse
\textmd \mdseries medium
\textbf \bfseries gras
          Macros pour la famille
\textrm \rmfamily romain
\textsf \sffamily sans serif
\texttt \ttfamily typewriter

% Macro de type 2 : {\sffamily {\bfseries Qui}, sur la terre ?}
% Macro de type 1 : seulement ce qui suit : ex \textbf{that}

% Macro emphasis (pour mettre en relief) : \emph{truc a mettre en relief}

%paragraphes : 
\begin{flushright} % ou {\raggedleft bla bla bla \par (ou ligne vide)}
Paragraphe a droit style anglais. Paragraphe a droit style anglais. Paragraphe a droit style anglais. Paragraphe a droit style anglais. Paragraphe a droit style anglais. Paragraphe a droit style anglais. Paragraphe a droit style anglais. Paragraphe a droit style anglais.
\end{flushright}

\begin{flushleft} % ou {\raggedright bla bla bla \par (ou ligne vide)}
Paragraphe a gauche style anglais.Paragraphe a gauche style anglais.Paragraphe a gauche style anglais.Paragraphe a gauche style anglais.Paragraphe a gauche style anglais.Paragraphe a gauche style anglais.Paragraphe a gauche style anglais.Paragraphe a gauche style anglais.Paragraphe a gauche style anglais.Paragraphe a gauche style anglais.
\end{flushleft}

\begin{center}% ou {\centering bla bla bla \par (ou ligne vide)}
Paragraphe centré.Paragraphe centré.Paragraphe centré.Paragraphe centré.Paragraphe centré.Paragraphe centré.Paragraphe centré.Paragraphe centré.Paragraphe centré.Paragraphe centré.Paragraphe centré.Paragraphe centré.Paragraphe centré.Paragraphe centré.Paragraphe centré.Paragraphe centré.Paragraphe centré.Paragraphe centré.Paragraphe centré.
\end{center}

% \noindent pour enlever l'indentation en debut de paragraphe

% \vfill, \hfill 
exemple :
Super\hfill Super \hfill Super%
Super\vfill Super \vfill Super%

% exemple de liste : itemize, enumerate, description
%\begin{type_de_liste}
%  \item premier élément de la liste
%  \item deuxième élément de la liste
%    .
%    .
%    .
%  \item dernier élément de la liste
%\end{type_de_liste}
%

\begin{itemize}
\item item1 ;
\begin{itemize}
\item item11 ;
\item item12 ;
\item item13.
\end{itemize}
\item item2 ;
\item item3.
\end{itemize}

\begin{enumerate}
\item item1.
\item item2.
\item item3.
\end{enumerate}

\begin{description}
\item[Un] item1
\item[Deux] item2
\item[Trois et etc.] item3
\end{description}

%LaTEX connaît 7 niveaux de plan. Dans l’ordre d’importance on trouve \part, \chapter, \section, \subsection,\subsubsection, \paragraph et \subparagraph.

\part{partie}
\chapter{chapitre}
\section{section}
\subsection{sous section}
\paragraph{paragraph}
\subparagraph{sous paragraphe}

\part{partie}
\chapter{chapitre}
\section{section}
\subsection{sous section}
\paragraph{paragraph}
\subparagraph{sous paragraphe}

\part{partie}
\chapter{chapitre}
\section{section}
\subsection{sous section}
\paragraph{paragraph}
\subparagraph{sous paragraphe}

\section{Quelques présentations plus évoluées}
\subsection{Structuration des documents}
\subsubsection{Commandes pour le plan}

\section{Quelques présentations plus évoluées}
\subsection{Structuration des documents}
\subsubsection{Commandes pour le plan}

\section{Quelques présentations plus évoluées}
\subsection{Structuration des documents}
\subsubsection{Commandes pour le plan}

%                 En effet, un appel à la macro \tableofcontents suffit à faire tout le travail. Certaines présentations
%nécessitent deux compilations successives pour être correctes. La construction d’une table des matières fait partie de
%celles-ci. La première compilation construit un fichier auxiliaire dans lequel sont rangés les renseignements concernant
%les titres (énoncés et numéros de page) et la seconde compilation lit ce fichier pour construire effectivement la table des
%matières.


%      Pour cela, il suffit de placer la macro \label suivi d’un groupe donnant le mot-clé permettant de retrouver cette
%référence. Ensuite, la macro \ref suivi d’un groupe ayant le même mot-clé donnera le numéro de la structure référencée (titre, tableau, figure, équation) et la macro \pageref donnera son numéro de page. Pour arriver à ceci, L TEX écrit ces informations dans une fichier auxiliaire lors de la compilation et il est nécessaire de compiler le source une seconde fois pour que ces informations puissent être lues.


%     Ainsi, la phrase :
%    Les commandes relatives au plan peuvent être vues à la
%     section~\ref{commandeplan} page~\pageref{commandeplan}.
%sera composée comme suit :
%     Les commandes relatives au plan peuvent être vues à la section 5.1.1 page 24. (Vous pouvez vérifier, je suis certain
%du résultat puisque ce n’est pas moi qui ai écrit les numéros.)


% Les annexes sont introduites par la macro \appendix. 



\end{document}

