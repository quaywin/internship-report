\documentclass{article}
\usepackage[latin1]{inputenc}
\usepackage[frenchb]{babel}
\usepackage[pdftex]{graphicx} % to include pictures
\setlength{\parindent}{3cm} % indentation des paragraphes
\setlength{\parskip}{18pt plus4pt minus2pt} % dimension des sauts entre les paragraphes
%\addto\captionsfrench{%
%  \renewcommand{\listfigurename}{Nouveau nom}%
%  \renewcommand{\listtablename}{Nouveau nom}%
%}
\title{Rapport
	\thanks{A special form of footnote}}
\author{Who wrote this stuff?}
\date{01/06/1986}

\begin{document}

\maketitle{}

\pagebreak



\listoffigures  % table des figures
\pagebreak
\listoftables   % table des tableaux
\pagebreak
\tableofcontents{}
\pagebreak

\section{Section}
\subsection{sous section}
\paragraph{paragraph}
\begin{figure}[htp]
\centering
\includegraphics{marcouille.png}
\caption{Marcouille figure}\label{fig:marcouille.png}
\end{figure}
\paragraph{paragraph}
\paragraph{paragraph}
\paragraph{paragraph}
\subsection{sous section}
\subsection{sous section}
\subsection{sous section}
\paragraph{paragraph}
\subparagraph{sous paragraphe}

\section{section}
\subsection{sous section}
\paragraph{paragraph}
\subparagraph{sous paragraphe}

\section{section}
\subsection{sous section}
\paragraph{paragraph}
\subparagraph{sous paragraphe}

\section{Quelques présentations plus évoluées}
\subsection{Structuration des documents}
\subsubsection{Commandes pour le plan}

\section{Quelques présentations plus évoluées}
\subsection{Structuration des documents}
\subsubsection{Commandes pour le plan}

\section{Quelques présentations plus évoluées}
\subsection{Structuration des documents}
\subsubsection{Commandes pour le plan}

%                 En effet, un appel é la macro \tableofcontents suffit é faire tout le travail. Certaines présentations
%nécessitent deux compilations successives pour étre correctes. La construction d?une table des matiéres fait partie de
%celles-ci. La premiére compilation construit un fichier auxiliaire dans lequel sont rangés les renseignements concernant
%les titres (énoncés et numéros de page) et la seconde compilation lit ce fichier pour construire effectivement la table des
%matiéres.


%      Pour cela, il suffit de placer la macro \label suivi d?un groupe donnant le mot-clé permettant de retrouver cette
%référence. Ensuite, la macro \ref suivi d?un groupe ayant le méme mot-clé donnera le numéro de la structure référencée (titre, tableau, figure, équation) et la macro \pageref donnera son numéro de page. Pour arriver é ceci, L TEX écrit ces informations dans une fichier auxiliaire lors de la compilation et il est nécessaire de compiler le source une seconde fois pour que ces informations puissent étre lues.


%     Ainsi, la phrase :
%    Les commandes relatives au plan peuvent étre vues é la
%     section~\ref{commandeplan} page~\pageref{commandeplan}.
%sera composée comme suit :
%     Les commandes relatives au plan peuvent étre vues é la section 5.1.1 page 24. (Vous pouvez vérifier, je suis certain
%du résultat puisque ce n?est pas moi qui ai écrit les numéros.)


% Les annexes sont introduites par la macro \appendix. 



\end{document}

