\documentclass{article}
\usepackage[latin1]{inputenc}
\usepackage[frenchb]{babel}
\usepackage[pdftex]{graphicx} % to include pictures
\setlength{\parindent}{3cm} % indentation des paragraphes
\setlength{\parskip}{18pt plus4pt minus2pt} % dimension des sauts entre les paragraphes
%\addto\captionsfrench{%
%  \renewcommand{\listfigurename}{Nouveau nom}%
%  \renewcommand{\listtablename}{Nouveau nom}%
%}
\title{Rapport
	\footnote{\thanks{A special form of footnote}}}
\author{Who wrote this stuff?}
\date{01/06/1986}

\begin{document}

\includegraphics[scale=0.8]{Couv-rapport.pdf}

\pagebreak



\listoffigures  % table des figures
\pagebreak
\listoftables   % table des tableaux
\pagebreak
\tableofcontents{}
\pagebreak

\section{Introduction}
\subsection{}
\paragraph{paragraph}
\begin{figure}[htp]
\centering
\includegraphics{marcouille.png}
\caption{Marcouille figure}\label{fig:marcouille.png}
\end{figure}
\paragraph{paragraph}
\paragraph{paragraph}
\paragraph{paragraph}
\subsection{sous section}
\subsection{sous section}
\subsection{sous section}
\paragraph{paragraph}
\subparagraph{sous paragraphe}

\section{section}
\subsection{sous section}
\paragraph{paragraph}
\subparagraph{sous paragraphe}

\section{section}
\subsection{sous section}
\paragraph{paragraph}
\subparagraph{sous paragraphe}

\section{Quelques pr�sentations plus �volu�es}
\subsection{Structuration des documents}
\subsubsection{Commandes pour le plan}

\section{Quelques pr�sentations plus �volu�es}
\subsection{Structuration des documents}
\subsubsection{Commandes pour le plan}

\section{Quelques pr�sentations plus �volu�es}
\subsection{Structuration des documents}
\subsubsection{Commandes pour le plan}

%                 En effet, un appel � la macro \tableofcontents suffit � faire tout le travail. Certaines pr�sentations
%n�cessitent deux compilations successives pour �tre correctes. La construction d?une table des mati�res fait partie de
%celles-ci. La premi�re compilation construit un fichier auxiliaire dans lequel sont rang�s les renseignements concernant
%les titres (�nonc�s et num�ros de page) et la seconde compilation lit ce fichier pour construire effectivement la table des
%mati�res.


%      Pour cela, il suffit de placer la macro \label suivi d?un groupe donnant le mot-cl� permettant de retrouver cette
%r�f�rence. Ensuite, la macro \ref suivi d?un groupe ayant le m�me mot-cl� donnera le num�ro de la structure r�f�renc�e (titre, tableau, figure, �quation) et la macro \pageref donnera son num�ro de page. Pour arriver � ceci, L TEX �crit ces informations dans une fichier auxiliaire lors de la compilation et il est n�cessaire de compiler le source une seconde fois pour que ces informations puissent �tre lues.


%     Ainsi, la phrase :
%    Les commandes relatives au plan peuvent �tre vues � la
%     section~\ref{commandeplan} page~\pageref{commandeplan}.
%sera compos�e comme suit :
%     Les commandes relatives au plan peuvent �tre vues � la section 5.1.1 page 24. (Vous pouvez v�rifier, je suis certain
%du r�sultat puisque ce n?est pas moi qui ai �crit les num�ros.)


% Les annexes sont introduites par la macro \appendix. 



\end{document}

