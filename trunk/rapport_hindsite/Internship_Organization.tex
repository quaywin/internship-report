\chapter{Internship Organization}

\section{Objectives of the internship}

In this section I will try to detail the real objectives of my final degree project.
Firstly the main goal was to continue the Easy Web Content project
initiated a few years ago at Hindsite Interactive.
Previous interns started to develop an image editor for the platform Easy Web Content using the Adobe
Flash technology and HTML/AJAX. This image editor was designed in Flash CS4 and used actionscript 3 code source for the logic.
This leads to the first task accomplished during my internship, the finalization
 of a new Flash Image Editor.
 
 Then, I worked on the main part of my internship; the development of a site builder to complete the Easy Web Content platform. This project started from scratch and I worked on it with Thomas Louis. We also had to collaborate with designers for the user interface and the "look and feel" aspect.

One month and half befor the end of my internship I began to work on several projects for the clients of Hindsite Interactive. During the next weeks I was assigned more complex tasks which I will talk in the following chapters.

\section{Agenda of the Internship}

The table \ref{tableau:agenda} presents the different tasks I was assigned during my internship.

\begin{table}[!h]
	\caption{\label{tableau:agenda}Internship Agenda}
	\begin{tabular}{ | l | p{12cm} | }
		\hline
		 & Tasks\\
		\hline
		Week 1	&	Population of content for Sencore's website\\	\hline
		Week 2	&	Development of Sencore's Website. Read documentation of the different projects of Hindsite Interactive. Get comfortable with the main project Easy Web content.\\	\hline
		Week 3 to 4	&	Work on Easy Web Content's Image Editor. Issues fixes.\\	\hline
		Week 5 to 7	&	Work both on the Image Editor and the specifications of the new Easy Web Content's Site Builder. Implementation of Performance Improvement tools.\\	\hline
		Week 8 to 12	&	Brainstorming and resarches about the Site builder features. Study the conpetitors. MVC Framework development.\\	\hline
		Week 12 to 16	&	Application of web design on the tools of the Site builder. Database development, drag and drop. User registration. Database automated scripts.	\\	\hline
		Week 17	to 18&	Work on Site Builder's widget features. Implementation of Video player feature on Sencore project. Work on secured payment form on A2LA project. Begin to communicate with clients.\\	\hline
		Week 19	to 22 &	Work on a mailling form for the client Advanced Centrifugals coupled with a backend. Bug correction on Sencore and Sentara projects. Work on the Site Builder in parallel\\	\hline
		Week 23	to 24&	Work on Pylon Project\\	\hline
	\end{tabular}
\end{table}

\section{Methodological aspect}

Hindsite Interactive possess more than 10 year in web design and development. During these years 
some methodological standards has been established. A standard project for a client takes place the following way.

\subsection*{Specification of the needs}
This step is a discussion with the customer to understand his expectations
and needs as well as establish a quote for the job and determine the duration
of the development.
This phase will for instance estimate the number of pages, the design work (if
the users wants Flash animations, unique fonts, a new logo, etc.), the degree
of administration possibilities (static HTML, partially manageable, or a full
CMS).

This step is supervised by Mr. Taei, who consults designers or developers to
get an accurate idea of the cost, time and possibilities.

\subsection*{Presentation of one or many prototype}
The next step consists of a presentation of a static design of the future web
site layout. Many different designs can be showed to the client to allow him to
choose the one that better fits his expectations.
The design is done by the design team, most of the time by Mr. Kheradmand.

\subsection*{Conversion of the design to HTML and CSS}
This phase is made up of the slicing of design elements (backgrounds,
images, titles, etc.) to images and of the conversion to a static design into the
HTML/CSS website. All the text content is static at this phase.

\subsection*{The development}
During the next phase, the dynamic pages are written, all the pages which
content need to be editable by the client are integrated in an administration
area. This is the phase that most concerns me and other coworkers of the
development team.

\subsection*{Presentation and Validation}
The website (at this moment hosted on Hindsite’s servers) is presented to the
client for validation. We give him an access to the administration area to test
the different functionalities and we provide him explanation of how to use it if
necessary.

\subsection*{Deployment}
When the website is approved, we move on to the deployment on either client
hosting server or our server with the definitive DNS redirection. We also use a
clean database from all the tests.
