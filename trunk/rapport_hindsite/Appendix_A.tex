\chapter{Site Builder Competitors}\label{annexe:ews-competitors}
\begin{figure}[!ht]
\centering
\includegraphics[width=.70\textwidth]{img/weebly.png}
\caption{Weebly}
\label{figure:weebly}
\end{figure}

\begin{figure}[!ht]
\centering
\includegraphics[width=.70\textwidth]{img/drupal_gardens.png}
\caption{Drupal Gardens}
\label{figure:drupal_gardens}
\end{figure}

\begin{figure}[!ht]
\centering
\includegraphics[width=.70\textwidth]{img/basekit.png}
\caption{BaseKit}
\label{figure:basekit}
\end{figure}

\begin{figure}[!ht]
\centering
\includegraphics[width=.70\textwidth]{img/squarespace.png}
\caption{Squarespace}
\label{figure:squarespace}
\end{figure}