\chapter*{Internship Organization}

%\section{Objectives of the internship}


%The initial internship subject submitted by the company was “”.
%In this section I will try to detail the real objectives of my final degree project.
%Firstly the main goal was to continue the Easy Web Content Add-ons project
%initiated a few years ago at Hindsite Interactive.
%Previous interns started to develop a few number of add-ons using the Adobe
%Flash technology. At the time I started my internship there were a Photo
%Gallery, a Music Player and a NewsTicker. All these three add-ons were
%administrated in what we call the “admin area”, an interface to customize, edit,
%populate and publish the Flash add-ons.
%This leads to the first task accomplished during my internship, the complete
%realization of a new Flash add-on compatible with the existing manager, the
%“Flash Maker”.
%
%Flash-maker add-on
%
%This add-on could be compared to a slide-show: a client/user can create
%different clips with their own image and texts (a title and a description) and use
%animated effects to transition between many clips. This add-on could for
%example be used as a company or a product presentation on the homepage of
%a client website.
%My objective was to create this add-on using the Flash technology and Action
%Script 3.0. The client data has to be stored in a database and the images are
%hosted on Hindsite’s servers. The add-on needs to be fully customizable by
%the client using the existing admin area for flash add-ons. I had to observe the
%specifications regarding the elements that can be customized by the user.
%
%The next add-on that I have to take care of is a calendar with the possibility to
%add events. We made not to use Flash technology to create this add-on but
%this decision also implied to create a new administration structure for the non-
%flash add-on (commonly called Html add-ons).
%
%WEINZAEPFEL Pierre
%
%The development of the admin section as well as the calendar was the next
%objectives of my internship.


\section*{Agenda of the Internship}

\begin{table}[!ht]
	\caption{\label{tableau:agenda}Internship Agenda}
	\begin{tabular}{ | l | p{12cm} | }
		\hline
		 & Tasks\\
		\hline
		Week 1	      &		Getting acquainted with Easy Web Content, reading architecture documentations\\	\hline
		Week 2 and 3  &		Debug of Easy Web Content Addon feature.\\	\hline
		Week 4 and 5  &		Work on Sencore Project: addition of upload, print page and form features.\\	\hline
		Week 6        &		Population of the Sencore products.\\	\hline
		Week 7  to 12 &		Brainstorming and researches about the Site builder features. Study and testing of the competitors. MVC Framework development.\\	\hline
		Week 13 to 16 &		Application of web design on the tools of the Site builder. Database development, drag and drop. User registration. Database automated scripts.	\\	\hline
		Week 17	to 18 &		Work on Site Builder's widget features (twitter, facebook, share, flickr, ...).\\	\hline
		Week 19	to 21 &		Work on page styling with css file creation and intuitive edition\\	\hline
		Week 22		  &		Work on page and menu template presets for the sitebuilder\\	\hline
		Week 23	to 24 &		Site builder page creation, file uploader and site publication feature\\	\hline
	\end{tabular}
\end{table}

%\section{Methodological aspect}

%The different steps of a website development at Hindsite are:
%
%- Specification of the needs
%This step is a discussion with the customer to understand his expectations
%and needs as well as establish a quote for the job and determine the duration
%of the development.
%This phase will for instance estimate the number of pages, the design work (if
%the users wants Flash animations, unique fonts, a new logo, etc.), the degree
%of administration possibilities (static HTML, partially manageable, or a full
%CMS).
%
%This step is supervised by Mr. Taei, who consults designers or developers to
%get an accurate idea of the cost, time and possibilities.
%
%- Presentation of one or many prototype
%The next step consists of a presentation of a static design of the future web
%site layout. Many different designs can be showed to the client to allow him to
%choose the one that better fits his expectations.
%The design is done by the design team, most of the time by Mr. Kheradmand.
%
%- Conversion of the design to HTML and CSS
%This phase is made up of the slicing of design elements (backgrounds,
%images, titles, etc.) to images and of the conversion to a static design into the
%HTML/CSS website. All the text content is static at this phase.
%
%WEINZAEPFEL Pierre
%
%This step is usually affected to Mrs. Hang Le, but I happen to give a hand for
%some CSS tricks or JavaScript animations/modules.
%
%- The development, add the Content Management System.
%During the next phase, the dynamic pages are written, all the pages which
%content need to be editable by the client are integrated in the administration
%area. This is the phase that most concerns me and other coworkers of the
%development team.
%
%- Presentation and Validation
%The website (at this moment hosted on Hindsite’s servers) is presented to the
%client for validation. We give him an access to the administration area to test
%the different functionalities and we provide him explanation of how to use it if
%necessary.
%
%- Deployment
%When the website is approved, we move on to the deployment on either client
%hosting server or our server with the definitive DNS redirection. We also use a
%clean database from all the tests.