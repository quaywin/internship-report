\documentclass[12pt,a4paper,utf8x,twoside,openright]{report}
\usepackage [english]{babel}

% Pour pouvoir utiliser 
\usepackage{ucs}
\usepackage[utf8x]{inputenc}



\usepackage{pifont}
\newcommand{\tick}{\ding{51}} % to write ticks
\newcommand{\badtick}{\ding{55}} % to write cross

\usepackage[pdftex]{graphicx} % to include pictures
\usepackage{pdfpages} % to include PDF pages as pictures (1st and last page)

\usepackage{url} % Pour avoir de belles url
\usepackage[top=2.5cm, bottom=2.5cm, left=2.5cm, right=2.5cm]{geometry}

\usepackage{wrapfig}
\usepackage{subfig}

\usepackage{color}
\usepackage{xcolor}

\usepackage{float}

% Pour mettre du code source
\usepackage {listings}
% Pour pouvoir passer en paysage
\usepackage{lscape}

\usepackage{caption}
\DeclareCaptionFont{white}{\color{white}}
\DeclareCaptionFormat{listing}{\colorbox{gray}{\parbox{\textwidth}{#1#2#3}}}
\captionsetup[lstlisting]{format=listing,labelfont=white,textfont=white}

\lstdefinelanguage{JavaScript}{
     keywords={attributes, class, classend, do, empty, endif, endwhile, fail, function, functionend, if, implements, in, inherit, inout, not, of, operations, out, return, set, then, types, while, use},
     keywordstyle=\color{blue}\bfseries,
     ndkeywords={},
     ndkeywordstyle=\color{yellow}\bfseries,
     identifierstyle=\color{black},
     sensitive=false,
     comment=[l]{//},
     commentstyle=\color{green}\ttfamily,
     stringstyle=\color{red}\ttfamily
  }

  \usepackage{courier}
 \lstset{
         basicstyle=\footnotesize\ttfamily, % Standardschrift
         %numbers=left,               % Ort der Zeilennummern
         numberstyle=\tiny,          % Stil der Zeilennummern
         %stepnumber=2,               % Abstand zwischen den Zeilennummern
         numbersep=5pt,              % Abstand der Nummern zum Text
         tabsize=2,                  % Groesse von Tabs
         extendedchars=true,         %
         breaklines=true,            % Zeilen werden Umgebrochen
         keywordstyle=\color{red},
    		frame=b,      	
    	 identifierstyle=\ttfamily,   
		 keywordstyle=\color[rgb]{0,0,1},
         commentstyle=\color[rgb]{0.133,0.545,0.133},
         stringstyle=\color[rgb]{0.627,0.126,0.941},
         %stringstyle=\color{white}\ttfamily, % Farbe der String
         showspaces=false,           % Leerzeichen anzeigen ?
         showtabs=false,             % Tabs anzeigen ?
         xleftmargin=17pt,
         framexleftmargin=17pt,
         framexrightmargin=5pt,
         framexbottommargin=4pt,
         %backgroundcolor=\color{lightgray},
         showstringspaces=false      % Leerzeichen in Strings anzeigen ?        
 }
 
\lstloadlanguages{XML, HTML,Java,PHP,SQL,JavaScript,bash,Ant}

% Pour pouvoir faire plusieurs colonnes
\usepackage {multicol}
% POur crééer un index
\usepackage{makeidx}
\makeindex

% Pour les entetes de page
\usepackage{fancyhdr} % Header and footnotes
\pagestyle{plain}


% Pour l'interligne de 1.5
\usepackage {setspace}

\parskip=5pt %% distance entre § (paragraphe)
\sloppy %% respecter toujours la marge de droite 

% Pour les pénalités :
\interfootnotelinepenalty=150 %note de bas de page
\widowpenalty=150 %% veuves et orphelines
\clubpenalty=150 

%Pour la longueur de l'indentation des paragraphes
\setlength{\parindent}{15mm}

%%%% debut macro pour enlever le nom chapitre %%%%
\makeatletter
\def\@makechapterhead#1{%
  \vspace*{10\p@}%
  {\parindent \z@ \raggedright \normalfont
    \interlinepenalty\@M
    \ifnum \c@secnumdepth >\m@ne
        \Huge\bfseries \thechapter\quad
    \fi
    \Huge \bfseries #1\par\nobreak
    \vskip 40\p@
  }}

\def\@makeschapterhead#1{%
  \vspace*{10\p@}%
  {\parindent \z@ \raggedright
    \normalfont
    \interlinepenalty\@M
    \Huge \bfseries  #1\par\nobreak
    \vskip 40\p@
  }}
\makeatother
%%%% fin macro %%%%

\setlength{\skip\footins}{1cm}

%Couverture 
% Le plus simple est d'importer un pdf s'il y a une couverture type a respecter
%\includepdf[pages=1,noautoscale=false]{premiere_page.pdf} 
\title
{
	\normalsize{Graduation Project}\\Promo 2010\\
	\vspace{15mm}
	\Huge{Junior Web Developer}
}
\author{Thomas LOUIS\\
	\vspace{45mm}
}

\date{	
	\normalsize{Lieu du stage\\
	Adresse du stage\\
	Ville du stage\\ 
	\vspace{5mm}	
	Directeur de recherche : M. DUPONT \\
	Rapporteur universitaire : Mme DUPUIS
	}
}

\begin{document}

\maketitle

\chapter*{Acknowledgements}

First I would like to thank Mr. TAEI Payman, Director of Operations and President of
Hindsite Interactive for allowing me to integrate this structure and have this unique
opportunity to realize my final degree project in the United States.
In particular I would like to thank Mr. TAEI for the confidence he granted to me to
lead the different projects I was affected to by considering me as an employee.
As my tutor I want to thank him for proposing me interesting, diversified and original
projects and tasks from a technical point of view. 

I especially enjoyed the autonomy
and responsibilities he entrusted me. But also follow-up, availability, advice and
criticism that enabled me to make progress and improve at the technical point of
view as well as methodological.
As well thank you to Mr. KHERADMAND Shayan, responsible of design who was
one of my co-worker and contact for all questions regarding the design. He was
always available and relevant in his answers.

I also would like to thank Mr. WEINZAPFAEL Pierre, ex-intern at Hindsite extending his
experience in the company for other six-months as a Web Developer. He was my
main co-worker and a precious help to understand the architecture and existing
code. Our mutual help allowed us to achieve a working project and a quick
progression of tasks.

Finally I thank all other employees of Hindsite at a distance or not for their patience
and help when needed.
%\clearpage

\tableofcontents
\listoffigures
%\listoftables
\clearpage

% Pour avoir un interligne de 1,5
\begin{onehalfspace}

\chapter*{Introduction}
\addcontentsline{toc}{chapter}{Introduction}

During my last year of studies at the University of Technology of Belfort-
Montbeliard (UTBM) in Computer Science Engineering, I acquired the
theoretical fundamentals in computer engineering, more particularly in
Software and Knowledge Engineering. In order to achieve my master in
engineering degree, my curriculum requires a 24-weeks final degree project to
apply and develop my skills in a professional activity.

I have chosen to apply to do this internship at Hindsite Interactive in order to
go into details in the web development and multimedia fields of study, an area
in expansion that interests me more particularly. In addition I wanted to do this
project in a foreign country to obtain an international experience. The project that
Hindsite offered me was ambitious and required the mastery of a range of
different technologies.

Hindsite Interactive is a small business specialized in multimedia, web
development, web design and website hosting. The main services proposed
are designing, web site creation and an online tool that can be compared to a
Content Management System (CMS) called Easy Web Content (EWC). This
tool allows customers to create a new website, edit an existing one and
manage their sites.

My internship as a junior web developer has for main task to construct a site builder for EWC and to develop widgets tht can be customize, edited and populated into web pages. An widget is defined as an interactive
dynamic module that can be inserted in the pages created with the site builder. The goal is
to allow people with no HTML/CSS or web creation knowledge to be able to easily create a web site.

After a presentation of the company and the Easy Web Content site, I will
describe the organization of the internship and finally detail the achieved work.

\clearpage


%samples chapter
%\chapter{Le titre du chapitre}

\section{Le titre de la section qui va bien}

\subsection{Titre de la sous section}

Exemple de code .
\lstset{language=HTML}
\begin{lstlisting}[label=some-css,caption=Some Html]
<style>
.header {
	background:#333;
	height:123px;

}
</style>

\end{lstlisting}
\lstset{language=PHP}
\begin{lstlisting}[label=some-php,caption=Some Php]
<?php
public class maClasse{
	public maFonction()
	{
		$maVar = new String("hello world");	
	}
}
?>

\end{lstlisting}
%$
\lstset{language=JavaScript}
\begin{lstlisting}[label=some-js,caption=Some JavaScript]

var mavar = new String('polop');
alert("hello world");
return false;

\end{lstlisting}

\lstset{language=SQL}
\begin{lstlisting}[label=some-sql,caption=Some SQL]

SELECT * FROM page WHERE id='1';

\end{lstlisting}

%-- Note de bas de page sur les stades
\protect\footnote{Par exemple, on peut faire un pied de page :
\begin{itemize}
\item avec une liste à puces ;
\item avec une liste à puces ;
\item avec une liste à puces.
\end{itemize}
}
%-- Fin Note de bas de page sur les stades

Ici du texte et du blabla, ce que l'on veut dire et écrire. A remplacer. Ici du texte et du blabla, ce que l'on veut dire et écrire. A remplacer. Ici du texte et du blabla, ce que l'on veut dire et écrire. A remplacer. Ici du texte et du blabla, ce que l'on veut dire et écrire. A remplacer. Ici du texte et du blabla, ce que l'on veut dire et écrire. A remplacer. Ici du texte et du blabla, ce que l'on veut dire et écrire. A remplacer.

\begin{itemize}
\item avec une liste à puces ;
\item avec une liste à puces ;
\item avec une liste à puces.
\end{itemize}

Ici du texte et du blabla, ce que l'on veut dire et écrire. A remplacer. Ici du texte et du blabla, ce que l'on veut dire et écrire. A remplacer. Ici du texte et du blabla, ce que l'on veut dire et écrire. A remplacer. Ici du texte et du blabla, ce que l'on veut dire et écrire. A remplacer. Ici du texte et du blabla, ce que l'on veut dire et écrire. A remplacer. Ici du texte et du blabla, ce que l'on veut dire et écrire. A remplacer.

\subsubsection{Titre de la sous sous section}

Ici du texte et du blabla, ce que l'on veut dire et écrire. A remplacer. Ici du texte et du blabla, ce que l'on veut dire et écrire. A remplacer. Ici du texte et du blabla, ce que l'on veut dire et écrire. A remplacer. Ici du texte et du blabla, ce que l'on veut dire et écrire. A remplacer. Ici du texte et du blabla, ce que l'on veut dire et écrire. A remplacer. Ici du texte et du blabla, ce que l'on veut dire et écrire. A remplacer.

%exemple d'image
\begin{figure}[!ht]
\centering
\includegraphics[width=.55\textwidth]{img/SP_A0183.jpg}
\caption{Image sample}
\label{figure:sampe}
\end{figure}

Ici du texte et du blabla, ce que l'on veut dire et écrire. A remplacer. Ici du texte et du blabla, ce que l'on veut dire et écrire. A remplacer. Ici du texte et du blabla, ce que l'on veut dire et écrire. A remplacer. Ici du texte et du blabla, ce que l'on veut dire et écrire. A remplacer. Ici du texte et du blabla, ce que l'on veut dire et écrire. A remplacer. Ici du texte et du blabla, ce que l'on veut dire et écrire. A remplacer.

\subsubsection{Titre de la sous sous section}

Exemple de tableau
\begin{table}[!ht]
	\caption{\label{tableau:evolPratXP}Evolution des pratiques XP au cours du stage}
	\begin{tabular}{|l|c|c|}
		\hline
		Pratique & Début du stage & Fin du stage\\
		\hline
		Pair programming & \tick & \tick \\
		Itération & 2 semaines & 1 semaine \\
		Intégration continue & \tick & \tick \\
		Niko Niko & \tick & \badtick \\
		Lecture & \tick & \badtick \\
		Point perso & hebdomadaire & nouvelles experimentations\\
		Pomodoro & \badtick & \tick \\
		Test driven development & \tick & \tick \\
		``Done done'' & théorique & adopté et normalisé\\
		``No Bugs'' & imprécis & engagement \\
		Slack Time & \badtick & adopté et normalisé\\
		Rétrospective & 1 à 2h le lundi matin & Timeboxée et juste après la livraison\\
		Point technique & assez réguliers & moins nombreux\\
		Veilleur & \tick & Robin cumule son rôle\\
		Batman/Robin & \badtick & \tick \\
		\hline
	\end{tabular}
\end{table}

\subsection{Conclusion}

Ici du texte et du blabla, ce que l'on veut dire et écrire. A remplacer. Ici du texte et du blabla, ce que l'on veut dire et écrire. A remplacer. Ici du texte et du blabla, ce que l'on veut dire et écrire. A remplacer. Ici du texte et du blabla, ce que l'on veut dire et écrire. A remplacer. Ici du texte et du blabla, ce que l'on veut dire et écrire. A remplacer. Ici du texte et du blabla, ce que l'on veut dire et écrire. A remplacer.

Ici du texte et du blabla, ce que l'on veut dire et écrire. A remplacer. Ici du texte et du blabla, ce que l'on veut dire et écrire. A remplacer. Ici du texte et du blabla, ce que l'on veut dire et écrire. A remplacer. Ici du texte et du blabla, ce que l'on veut dire et écrire. A remplacer. Ici du texte et du blabla, ce que l'on veut dire et écrire. A remplacer. Ici du texte et du blabla, ce que l'on veut dire et écrire. A remplacer.

\subsection{Titre de la sous section}

Ici du texte et du blabla, ce que l'on veut dire et écrire. A remplacer. Ici du texte et du blabla, ce que l'on veut dire et écrire. On peut faire une citation \cite{Motclef1}.
A remplacer. Ici du texte et du blabla, ce que l'on veut dire et écrire. A remplacer. Ici du texte et du blabla, ce que l'on veut dire et écrire. A remplacer. Ici du texte et du blabla, ce que l'on veut dire et écrire. A remplacer. Ici du texte et du blabla, ce que l'on veut dire et écrire. A remplacer.

Ici du texte et du blabla, ce que l'on veut dire et écrire. A remplacer. Ici du texte et du blabla, ce que l'on veut dire et écrire. A remplacer.
Ici du texte et du blabla, ce que l'on veut dire et écrire. A remplacer. Ici du texte et du blabla, ce que l'on veut dire et écrire. A remplacer. Ici du texte et du blabla, ce que l'on veut dire et écrire. A remplacer. Ici du texte et du blabla, ce que l'on veut dire et écrire. A remplacer.

\subsection{Titre de la sous section}

Ici du texte et du blabla, ce que l'on veut dire et écrire. A remplacer. Ici du texte et du blabla, ce que l'on veut dire et écrire. On peut faire une citation \cite{Motclef1}.
A remplacer. Ici du texte et du blabla, ce que l'on veut dire et écrire. A remplacer. Ici du texte et du blabla, ce que l'on veut dire et écrire. A remplacer. Ici du texte et du blabla, ce que l'on veut dire et écrire. A remplacer. Ici du texte et du blabla, ce que l'on veut dire et écrire. A remplacer.

Ici du texte et du blabla, ce que l'on veut dire et écrire. A remplacer. Ici du texte et du blabla, ce que l'on veut dire et écrire. A remplacer.
Ici du texte et du blabla, ce que l'on veut dire et écrire. A remplacer. Ici du texte et du blabla, ce que l'on veut dire et écrire. A remplacer. Ici du texte et du blabla, ce que l'on veut dire et écrire. A remplacer. Ici du texte et du blabla, ce que l'on veut dire et écrire. A remplacer.

\clearpage


%productivity tools
%projects

\chapter{Company presentation}

\section{Presentation}

Hindsite Interactive is a small-size enterprise located in Frederick, Maryland at
approximately 45 minutes to the large city of Washington DC. Hindsite owns a
subsidiary Webnet Hosting based in Rockville, at the periphery of Washington
DC. Both Frederick and Rockville are under the area of effect of the capital
city whose total agglomeration counts more than 5 billion inhabitants.

Headquarter:
Hindsite Interactive Inc.
241 East 4th Street
Frederick, Maryland 21701 USA

\subsection*{Short History}

Hindsite Interactive is a young company founded in 2000 by the actual
President of Operation, Mr. Payman Taei. Everything started from the corner
of a small apartment, \$170 and a vintage Pentium processor and is now a
business with two divisions.

As written on Hindsite Interactive website: his main goal is not to form a
conglomerate corporation but to provide a quality service and to maintain a
strong relationship with each of his client.

\section{Organizational Unit}

Hindsite is a very small company, it counts in total between 10 and 12
employees and they are not all working in the same place.

The following diagram is a representation of the organization unit at Hindsite
Interactive.

My role at Hindsite Interactive takes place in the development team, as a
junior web developer. My hierarchic superior is Mr. Payman Taei, who assign
my tasks most of the time. But I am also working closely from the designer Mr.
Shayan Kheradmand to take over the design phase with the development
one.
During all my internship I was working in collaboration with Mr. Michael
Valadier, web-developer and ex UTBM student.
Finally I have always been in touch with other developers at Hindsite who are
working from a different geographic place.

\section{Company activities}

Hindsite Interactive proposes different services though its different labels.
Around the Hindsite Interactive name are gathered:
- The web design and web development
- Graphic design and interfaces
- 3D modeling
- Animation and interactivity

The website development is diversified but essentially centered on the CMS,
e-commerce and dynamic applications.

Webnet Hosting

This subsidiary from Hindsite Interactive takes care of the management and
hosting of websites. In addition, the company offers a large number of
services that are a success with not less than a thousand websites hosted.

Overview of the main services and characteristics:
- Easy Web Content: the website editor
- A CMS: Site Studio Design that proposes 600 templates
- Urchin for analysis and statistics
- Word Press for the realization of forums, blogs, pools etc.
- CPanel, for site configuration and management
- Full backup every night

Since his creation Hindsite proposes solutions in term of design and animation
that use the newest technologies and therefore has a recognized know-how in
his field of competence.
A proof is Hinsite’s website entirely realized in Flash and animated. This site is
a portfolio itself of the company competences.
The company received awards for his website and the quality of work from
“PC World”, “Best of the Web”, “Perfectory Award” and the “Golden Web
Award” by The International Association of Web Masters and Designers.

4. Clients

Even Hindsite is still a young company, it owns already an extended contact
list. Here are some example of customer and projects:

ABC News: is a division of the famous television channel ABC, one of the
four top networks in the US. ABC News is a national news channel. Hindsite
realized a module for the website to cover the last presidential campaign.

WEINZAEPFEL Pierre

7

Avalere: is a leading advisory company focused on healthcare business
strategy and public policy. Our company realizes different projects for
Avalere’s customers.

Noland Aviation is an aircraft sales and acquisition company. Hindsite
Interactive entirely designed and developed the website and Flash animations.

Sentara College of Health Sciences is a university located in Virginia. It
belongs to the Sentara Health System organization that counts a hundred of
sites. Hindsite designed a website with a full CMS.

BABCN for Bay Area Biomedical Consultants Network, it provides
networking and educational opportunities for consultants in the San Francisco
life sciences community. Another example of a website developed by Hindsite
Interactive and in which I participated in the development.

Through these examples we see that clients come from various fields like
industry, services, consulting, education, health, Medias, etc.
This shows that Hindsite knows how to adapt to the specific needs of the
different clients.

5. Services proposed

a. Complete website development

Hindsite proposes its clients to create their websites from A to Z, including the
specification, the design, the development, eventually an administration
section and hosting on its server.

WEINZAEPFEL Pierre

8

b. Other design projects

This can be a CD/DVD presentation, a logo, a Flash animation or banner, a
company presentation in a video format, advertisements, etc.

c. Easy Web Content

Easy Web Content is a solution that can be compared to a CMS (Content
Management System). This innovative web platform allows website owners to
update easily and maintain their websites.

The solution proposed is entirely online and doesn’t require any download and
installation. As a matter of fact, the system is modular and flexible.
It is important to highlight that the system does not require any knowledge in web
development and everything is done to facilitate usability. Most of the users are
novice.

Easy Web Content allows users to manage their web pages, edit content, manager
their FTP server but also to add dynamic content to their pages as a music player, a
calendar etc.

Major possibilities of the software:

-
-
-
-

Edition of all pages and content
Creation of web pages
Addition of interactive, dynamic content in Flash and/or HTML (all add-ons)
Security of pages with a login/password access

6. Company Analyze

Hindsite Interactive has a strategic geographic position, indeed Frederick is
located inside an agglomeration with a worldwide influence in the eastern
megalopolis of the USA and its main city New York City. This area is the
economical heart of the country and counts more than 55 million inhabitants.

WEINZAEPFEL Pierre

9

It’s also one of the most influent in the world in the economical, financial,
political or even cultural aspects. All this favorites businesses of all kind and
especially requires new technologies and information systems (IT). As a
matter of fact, Hindsite Interactive occupies a central position that allows it to
answer quickly and efficiently to the customer needs.

Closer to Frederick, Washington DC is the capital of the USA and therefore
hosts thousands of federal institutions, organizations, law firms, medias,
consulting companies etc. Moreover the Washington area comprises
worldwide companies headquarters and some of the most famous universities
(e.g. Georgetown).

Hindsite knew how to take advantage of this heterogeneity by offering different
services and have client in all sectors. This non-specialization offers Hindsite
the ability to continue growing during the economical crisis.
In order to comply with the Business Model of the company, the philosophy is
to develop the interdisciplinary of competences and a quick adaptation to the
new technologies to answer the best to the client expectations.
\chapter{My first steps at HindSite, Inc. }

\section{Easy Web Content}

The first task I was given when arriving at HindSite Interactive was to get acquainted with their main application: Easy Web Content and especially with what is called the Easy Web Content Addons.

Easy Web Content is a web based software developed by HindSite Interactive. With it, users can easily manage their existing website: adding content, search engine optimizing, duplicate pages, ... Addons are specific complex content you can add to your pages : Photo Gallery, Calendar, Music Player, ... As the implementation of these addons was quite new in the software, some bugs were still existing. So I had to get familiar with the way they were built, to be able to resolve those bugs so that these addons could go live on the application. Previous interns had made a lot of documentation on them. It took me a week to read all the documentation and understand how they were built.

Also, the source code and user database information of these addons are located in another server than the one Easy Web Content runs in. So I also had to get familiar with the way the Easy Web Content server would communicate with the Addon Server to retrieve the information needed to handle these addons. That was pretty interesting as the communication uses a protocol I had never used before : SOAP \footnote{Simple Object Access Protocol}.


SOAP is a web service protocol that aims at exchanging structured information. It uses XML\footnote{eXtensible Markup Language} as its message format and RPC\footnote{Remote Procedure Call} or HTTP\footnote{HyperText Transfer Protocol} for its transmission. This protocol is based on three parts:
\begin{itemize}
\item The envelope, which defines what is in the message and how to process it
\item A set of encoding rules for expressing instances of application defined datatypes
\item A convention for representing procedure calls and responses
\end{itemize}
\lstset{language=XML}
\begin{lstlisting}[label=soap request,caption=Example of a SOAP request]
<?xml version="1.0"?>
<soap:Envelope
xmlns:soap="http://www.w3.org/2001/12/soap-envelope"
soap:encodingStyle="http://www.w3.org/2001/12/soap-encoding">

<soap:Body>
  <m:GetPrice xmlns:m="http://www.w3schools.com/prices">
    <m:Item>Apples</m:Item>
  </m:GetPrice>
</soap:Body>

</soap:Envelope>
\end{lstlisting}
\lstset{language=XML}
\begin{lstlisting}[label=soap response,caption=Example of a SOAP response]
<?xml version="1.0"?>
<soap:Envelope
xmlns:soap="http://www.w3.org/2001/12/soap-envelope"
soap:encodingStyle="http://www.w3.org/2001/12/soap-encoding">

<soap:Body>
  <m:GetPriceResponse xmlns:m="http://www.w3schools.com/prices">
    <m:Price>1.90</m:Price>
  </m:GetPriceResponse>
</soap:Body>

</soap:Envelope>
\end{lstlisting}
In this example, the client requested a web service called GetPrice, located at the url http://www.w3schools.com/prices. The SOAP server would execute the request, ie run the function GetPrice and return the response.

As it is XML based, SOAP is independent to whatever platform or programming language you are using. As an example, Easy Web Content server is using Java whereas the Addon server is using Php. Still, they are able to understand and handle the requests/responses that transit both ways.
The server that needs to retrieve the information is called a Soap Client. The one that provides the information is called the Soap Server. It provides what is called a Web Service. Any server around the world can retrieve the same information using this Web Service.

When I was fully acquainted with this protocol, I was able to create a new feature for Easy Web Content users: the creation of demo accounts. Users where now able to create demo accounts on Easy Web Site with the possibility to view addon presets so that they could get an overview of what they could do if they signed up for the complete version.


\section{A client project: Sencore}
After working on Easy Web Content, I was given the task to assist another developer in a huge client project for the company Sencore. The project's aim was to give the client a fully customized CMS\footnote{Content Management System} in which they could manage pretty much every aspect of their website, from customer requests, job applications to website content and product description. This project was my first real experience in Php/Mysql, and I was really excited about that.

The first task I achieved was the creation of a file uploader so that the client could upload their assets (images, pdf) to the website and then select from a list to incorporate them in the page.
\begin{figure}[!ht]
\centering
\includegraphics[width=.55\textwidth]{img/sencore.jpg}
\caption{Sencore File Uploaded}
\label{figure:sencore_uploader}
\end{figure}
\\The client can select from any image file on his computer and then upload it to the server. He will then directly see his image appear in the list underneath. Upon selection of the image, the popup will close and the image will be displayed in the appropriate content management section.

Then I created a "Print Page" functionality to allow customers to print the specification of a product. This task is basically achieved by changing the CSS of the html content you want to be printed: instead of setting width and padding of elements in pixels, you need to set it in percentage, a 100\% width meaning that the element will take the full width of a sheet of paper.

The last task I had to achieve in this project was not a development task. Unfortunately, sometimes the client will request you to do non development related stuff. This task consisted in populating all of their products in the Content management system, given pdf sheet they would give us. This was a really wearisome task as it meant doing the same thing over and over again for each of their products. I took me about a week to do so.

\chapter{Easy Web Content Site builder.}

The Easy Web Content Site Builder, also called EWS, is the main project we are working. This project intends to complete the existing Easy Web Content service by enabling the user to create a new website from scratch.
%descriptopn of EWC
Basically The site builder will make extensive usage of ajax. The main directions of the project are
\begin{itemize}
\item Be the most simple in terms of user experience
\item propose simple to create but adapted styles and easy to customise
\item does not have the drawbacks of the concurrence
\item have the advantages of the concurrence
\item be scalable
\item be securized
\item target users : 2/3 of them have no HTML/CSS knowledge
\end{itemize}

%show a sample of the concurrence

\begin{wrapfigure}{l}{0.5\textwidth}
  \begin{center}
    \includegraphics[width=.40\textwidth]{img/ews_archi_before.png}
  \end{center}
\caption{Basic EWS architecture }
\end{wrapfigure}

\begin{wrapfigure}{r}{0.5\textwidth}
  \begin{center}
    \includegraphics[width=.40\textwidth]{img/ews_archi_after.png}
  \end{center}
\caption{EWS architecture rethink for cloud environment}
\end{wrapfigure}

\begin{figure}
  \centering
  \subfloat[A gull]{\label{fig:gull}\includegraphics[width=0.4\textwidth]{img/ews_archi_before.png}}
  \hspace{10pt}               
  \subfloat[A tiger]{\label{fig:tiger}\includegraphics[width=0.4\textwidth]{img/ews_archi_after.png}}
  \caption{Pictures of animals}
  \label{fig:animals}
\end{figure}

So the first stpe into the project was to brainstorm about the feasibility of the software. We did several research of was already existed in the market.
Writing downs the strengths and weaknesses of the most famous site builders. At the same time we specified the structure of the information system which
would allow the site builder to be scalable, and modular. We try to figure what was the best choices between achitecture and programmation languages.
After a week we decided to adopt PHP MySQL and Javascript instead of j2EE , GWT. The developers into the team are more familiar with these technologies.
So we designed a relational model for the MySQL database that matched our first expectations in term of page creation and page organisation.
%several illustrations of the models evolution



To allow scalability security and future cloud architecture, we decided to split the application into several databases inside the same MySQL server.
Inside the same MySQL server the databases can communicate easily with each other. So it allows to decouple these databases.
\begin{itemize}
\item a database to store common data
\item a database to manage users and sites
\item one database per site 
\end{itemize}
To allow better security, at the creation of a new site database ze create a specific MySQL user for this database which have specific permissisons
only on that database. It can acces common database data thru VIEWS.
% databases architecture illustration
After several meetings we decided to adopt a system that stores the informations of each pages created by the editor into a datababase.And then when
the user decides he will publish the pages that are converted in plain php or html pages.

\section{Web issues}

To understand the issues that arises when developing a web site or web application, let me first remind you how a user can access a web page.
\begin{itemize}
\item First, the client (the user's web browser) requests a web page located on a web server. The function of this web server is to deliver the requested web page to the client. This means delivery of an HTML document along with additional content, 
such as images, stylesheets and JavaScripts.
\item The client's web browser application then processes the content and displays it for the user.  
\end{itemize}

\begin{figure}[!l]
\centering
\includegraphics[width=.55\textwidth]{img/static.png}
\caption{Case of a static Html page}
\label{figure:static page}
\end{figure}
\begin{figure}[!r]
\centering
\includegraphics[width=.55\textwidth]{img/static.png}
\caption{Case of a dynamic page (for example a php page)}
\label{figure:dynamic page}
\end{figure}

While the server will always deliver the same content for the same requested page, two different users might see two different things on their computer. 
The reason is that there are more than one web browser software application existing, and while they try to make it look the same, the process behind the display and interpretation of the content is not. Set aside Html content, which is quite processed in the same way among every browsers, issues arises with interpretation of
JavaScript and Stylsheets (CSS). For example, Firefox will understand the CSS property moz-border-radius while Google Chrome and Internet Explorer will not. Also, Firefox will not understand the JavaScript Object window.event while Google Chrome and Internet Explorer will. There are dozens of example like these ones.
While you can develop the server side part of the application in one language and not worry about its ability to always deliver the same content, you can't afford to develop the client side application for only one browser. This would result in many internet users not to be able to see the content as intended.
\begin{figure}[!ht]
\centering
\includegraphics[width=.55\textwidth]{img/browser_statistics.png}
\caption{Market Share of Web Browser}
\label{figure:Market Share of Web Browser}
\end{figure}
And as a web development company, that is even less of an option. (your clients won't like that if you tell them that only 30 of the internet users will be able to use their website.)
Web application that behaves the same way in multiple browser are called "cross-browser". 
My job was of course to make sure that every web content I created was cross-browser (at least for IE7, IE8, Chrome and FF).
At first, this can look like a hard task to accomplish, because you would need to be aware of each browser specificities. Fortunately, some tools exists to help you reach that goal.

\subsubsection{JavaScript}

JavaScript has some big differences between browsers. For example, to attach an event to an element, Firefox and Chrome uses the function addEventListener. Internet Explorer will not understand this function as it uses attachEvent for the same purpose. %(cf Appendix page ## http://www.quirksmode.org/dom/w3c\_CSS.html).
Also, one of the purpose of JavaScript is being able to handle all the html elements in the web page to increase the user experience on the website. And again, there are some slight differences between every browser.
Because of all those disparities, people started to build APIs, which would considerably reduce the knowledge needed of those disparities to run a cross-browser JavaScript code. On of the most famous is jQuery. Simple, yet powerful, you can find in jQuery all the JavaScript functionalities, encapsulated in cross-browser functions. Ex: instead of having to test if the browser is IE to call the function attachEvent instead of addEventListener, you can just call the jQuery method bind to attach an event to an element. No need to learn the supported functions of each browser, you just need to learn a simple API. jQuery even simplifies some JavaScript functionalities, such as the use of Ajax.
%(cf piece of code w w/o jQuery).

\subsubsection{CSS}

Unlike in JavaScript, there is no API to help you with CSS (mainly because CSS is not a programming language). The huge issues in CSS arises when you want to make it work on Internet explorer. Usually, you are going to develop with Chrome or Firefox (Internet Explorer can be very slow so it can be frustrating to develop on it) where the same CSS is going to work on each other. It might even work in Internet Explorer 8 (with some really little adjustment on precise cases). But most of the time, you are going to have to do some adjustments so that it works on Internet Explorer 7. Those adjustment might be to just rethink the way of styling an element so that the same CSS will work on IE7 or completely tweaking the CSS just for Internet Explorer.
The last is called "CSS hacking". And fortunately, probably because Microsoft knows that his CSS handling is not that good, it introduced what is called conditional comments. It looks like a simple html comments in the html content, but IE knows it is aimed at him and can process it to add a specific Stylesheet. Other browsers will just ignore it, as it is a mere comment for them. (cg example html + speak about prefix * \_ etc.). 
\begin{figure}[!r]
\centering
\includegraphics[width=.55\textwidth]{img/comments.png}
\caption{Conditional Comment}
\label{figure:conditional comment}
\end{figure}
Another way to tweak the CSS specifically for Internet Explorer is to add a specific prefix to each CSS property you want to be applied only to some Internet Explorer Versions. (prefix * : IE 7 and below,prefix \_: IE6 and below). The downside of this is that it renders the CSS invalid to the W3C specification.

Understanding all that, you can understand why creating a site builder application is even more of a challenge. Indeed, we have to provide the user that has no understanding of html, CSS or JavaScript the ability to create a cross-browser website. And that was an important point, because it could give us an advantage against other competitors, who are not always cross-browser compliant.




\chapter{Productivity tools}

\section{Pear}

PEAR is a framework and distribution system for reusable PHP components. It allows to easily install, remove and update php libraries and packages by using simple command line.
\\

\lstset{language=bash}
\begin{lstlisting}[label=pear-install,caption=Installation of pear packages]
C:\...>pear channel-discover channelname
C:\...>pear install --alldeps channelname/packagename
\end{lstlisting}

\section{Phing}

To work more efficently automatic build task could be used to improve productivity. Phing is a tool written in PHP. It is very similar to famous ANT.
Phing is a build tool. A phing build task has been set to automatize js/css minification with YUI Compress.
Right now we can use it to automate deployment and development.
\begin{itemize}
\item automatic js / css minify for release
\item automatic backup
\item automatic database deployment
\item generate php documentaion from comments
\end{itemize}

Phing uses pear to be installed automatically.

\lstset{language=bash}
\begin{lstlisting}[label=phing-install,caption=Installation of Phing]
C:\...>pear channel-discover channelname
C:\...>pear install --alldeps channelname/packagename
\end{lstlisting}

A build is configured with an xml file : build.xml. It allows to program automated tasks called targets. Target can be combined and ordered.
Obfuscate javascript or css file would have been a very repetitive and boring task to do manually. This tool allows to keep code unobfuscated for development environement (easy to debug and read), and obfuscated in production environment (lighter, faster and harder to read).

\lstset{language=Ant}
\begin{lstlisting}[label=phing-build,caption=Example of Phing build.xml]
<project name="EasyWebSite" basedir="." default="info">
	<property name="ProjectVersion" value="0.0.1" />
	<target name="minify-js">
		<minify targetDir="${releaseDirectory}/Framework/public/js/"
				yuiPath="${buildScriptsDirectory}/yuicompressor-2.4.2.jar">
			<fileset dir="${applicationDirectory}/Framework/public/js/">
			<include name="**/*.js"/>
			<exclude name="library/**"/>
			</fileset>
		</minify>
	</target>	
	<target name="zip-backup"
		description="Creates a backup of the project">
		<echo msg="Backup to zip archive"/>
		<copy file="build.xml" tofile="build.xml.backup" overwrite="true"/>
		<mkdir dir="${backupDirectory}" />				
		<zip destfile="${backupDirectory}/ews_backup-${ProjectDate}.zip">
		<fileset dir=".">
			<include name="build.xml" />
			<exclude name="build/**"/>
			<exclude name="docs/**"/>
			<include name="src/**" />			
		</fileset>
		</zip>
	</target>
	<target name="build-all" depends="cleanup , setup-environments, prepare-libs , execute-phpdoc, execute-jsdoc , deploying-debug, deploying-release, minify-js ,minify-css, deploy-database"
		description="deploys full environment">       
		<echo msg="Fin du build" />		
	</target>		
</project>
\end{lstlisting}



\section{Subversion}

Subversion is a version control system. It allows to keep track to modifications of files revision after revision. A system like this is very important to use in a team environment. For several reasons.

\paragraph*{Critic portions of code}
When working on important pieces of code the developer is often afraid to introduce bugs of to make the application crash. With a version control system, he can rollback to previous stable version of the code easily. So he can experiment new features with more confidence. 

\paragraph*{Debug}
If a bug is introduced, the developer can check what were the modifications in which files by using a diff tool to compare two revisions of a the same file.

\begin{figure}[!ht]
\centering
\includegraphics[width=.85\textwidth]{img/diff.png}
\caption{Diff tool}
\label{figure:diff}
\end{figure}

- compare sources between different revisions
- team work work on the very same files / not overwrites someone elses modifications

\section{Netbeans}
-code autocompletion
- code navigation
- code syntax recognition
- classes heritage detection


\chapter*{Conclusion}

\section*{Overview of the work achieved}

During this internship I learned a lot in web development. I had never done 
Php/Mysql in a professional environment and now it is safe to say that I am pretty 
comfortable with these technologies and the extent of it.
\\I was also able to improve my knowledge in client side language, such as HTML, CSS and 
JavaScript, which I had previously worked with in my previous internship. I also discovered 
and used the JavaScript API jQuery, which is one of the most used and popular API to develop with.

Some of the other fields I was able to discover were Web Services (SOAP), Flash (especially ActionScript), 
working with SVN and Mantis and build tasks (Phing).

I am now staying at Hindsite six other months and will continue developing mainly for the Easy 
Web Content Site Builder. We should soon begin a test phase that will give us an idea of how far 
we are in the project and if we can put down a launch date anytime soon.

\section*{Work and organization}

In the United States the hours of work per week are quite the same than in France (around
40 hours) and I was mostly satisfied with the progression of my work.
I was granted a lot of independence in the choice of technologies, implementation,
methods and organization. I am really glad I was able to start a new project as it involved 
a big part of conception and brainstorming, not only development. I could always give my opinion and submit my ideas, and 
choose which direction to take development-wise.

When I arrived at Hindsite Interactive, there was no project management tool, and I think Hindsite could work on that
to ease the development of new projects. 

For example we do not have any Subversion tool to work with and as we are often 
two people working on shared resources this was always an issue and 
it happens that we lose time because of that.
This is a point that me and previous interns already talked about to Mr. Taei but
because do not have access to the server configuration we cannot configure a SVN
ourselves.
\\Another improvement of organization would be to use a Bug Tracking tool. Right now we are still
using google documents for many of the projects and in those you can't really track the advancement of a project or debug easily.

We had those tools set up for the Easy Web Content Sitebuilder, 
and it would have been really hard to develop this application without them.

% Pour finir l'interligne de 1,5
\end{onehalfspace}

%----------------------------------------
% Pour la bibliographie
%----------------------------------------
% Citer tous les ouvrages/références
% pour ajouter des references : ouvrir biblio.bib
%nocite permet d'ajouter de la biblio sans avoir aciter 
%quelquechose dans le document
% si quelquechose a citer voir \cite
\nocite{MotClef1}
\nocite{MotClef2}
\nocite{MotClef3}
\nocite{MotClef4}
\nocite{MotClef5}
% Trier par ordre d'apparition
%bibliographystyle{unsrt}
% Pour le style de la biblio
\bibliographystyle{plain}
\bibliography{biblio}
\printindex
%exemples d'annexes
\appendix
%\chapter{Site Builder Competitors}\label{annexe:ews-competitors}
\begin{figure}[!ht]
\centering
\includegraphics[width=.70\textwidth]{img/weebly.png}
\caption{Weebly}
\label{figure:weebly}
\end{figure}

\begin{figure}[!ht]
\centering
\includegraphics[width=.70\textwidth]{img/drupal_gardens.png}
\caption{Drupal Gardens}
\label{figure:drupal_gardens}
\end{figure}

\begin{figure}[!ht]
\centering
\includegraphics[width=.70\textwidth]{img/basekit.png}
\caption{BaseKit}
\label{figure:basekit}
\end{figure}

\begin{figure}[!ht]
\centering
\includegraphics[width=.70\textwidth]{img/squarespace.png}
\caption{Squarespace}
\label{figure:squarespace}
\end{figure}
%\chapter{Cross browser issues}\label{annexe:JavaScript differences}


\section*{Example of javascript differences between browsers}
\begin{figure}[!ht]
\centering
\includegraphics[width=\textwidth]{img/css.jpg}
\caption{Css handling differences}
\label{figure:market-brosers}
\end{figure}


%\chapter{Pomodoro Technique}\label{annexe:pomodoro}

\paragraph*{Qu'est ce que c'est?}La technique du pomodoro a été inventée par Fransesco Cirillo. C'est une technique de gestion du temps qui peut être utilisée pour n'importe quelle type de tâche. L'objectif de la technique du pomodoro est de considérer le temps comme un allié dans ce que l'on veut faire et d'améliorer en permanence notre façon de travailler ou d'étudier.

\paragraph*{Que faut-il pour commencer?}
\subparagraph*{Une minuterie de cuisine}
Vous pouvez aussi bien utiliser un pomodoro\footnote{minuterie de cuisine} qu'un timer logiciel. Le régler sur 25 minutes.


% 4eme de couverture
%\includepdf[pages=1,noautoscale=false]{derniere_page.pdf}
\end{document}
