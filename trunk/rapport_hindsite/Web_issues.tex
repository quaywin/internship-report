Web issues.

 
To understand the issues that arises when developping a web site or web application, let me first remind you how a 
user (we will call him a client) can see a web page. First, the client requests a web page located on a web server. The function of this web server is 
to deliver the requested web page to the client. This means delivery of an HTML content along with additional content, 
such as images, stylesheets and javascripts. The client's web browser application then processes the content and displays it for the user. 
While the server will always deliver the same content for the same requested page, the user might see a different output. 
The reason is that there are more than one web browser software application existing, and while the output usually looks the same on any of them, the process behind 
displaying and interpreting the content is different. While Html content is quite processed in the same way among every browsers, issues arises with interpretation of
javascript and stylsheets. For example, Firefox can understand the css property moz-border-radius while Google Chrome and Internet Explorer will not. Also, Firefox 
will not understand the Javascript Object window.event while Google Chrome and Internet Explorer will.
While you can develop your server side part of the application in one language and not worry about its ability to always deliver the same content, you can't afford to
develop the client side application for only one browser. This would result in many internet users not to be able to see the content as intended.
% w3c table showing market share of web browser on the site of the page %
And as a web development company, that is not an option. (your clients won't like that if you tell them that only 30 % of the internet users will be able to use 
their website.)
Web application that behaves the same way in multiple browser are called "cross-browser". 
My job was of course to make sure that every web content I created was cross-browser (at least for IE7, IE8, Chrome and FF).
Fortunatly, some tools exists to help you reach that goal.
1/ Javascript
jQuery
2/ Css
conditionnal comments