\chapter{Conclusion}

\section{Synthèse des résultats}
\subsection{Difficultés rencontrées}
Les principale difficultés que j'ai rencontré ont été l'adaptation à l'environnement de travail sous linux et aux différentes technologies et techniques que je ne conaissais pas.
L'adaptation au travail en équipe a été un peu difficile au début du stage, avant de connaitre un peu mieux les membres de l'équipe et de communiquer plus facilement avec eux.
\section{Connaissances acquises}
\subsection{Programmation JAVA}
J'ai pu comprendre un peu plus précisément certaines notions de la programmation objet en Java comme la généricité ou encore quelque patron de conception tels que les visiteurs. Néanmoins j'ai appris également que beaucoup de patrons de conception necessitent plusieurs mois d'expérience afin d'être utilisés dans les bons cas.
\subsection{Design de code}
Habitué aux spécifications détaillées en amont avec des diagrammes UML, j'ai appris qu'il existait d'autres façon de structurer et de concevoir une application informatique. Le code etant lui-même sa documentation. Cela venant du fait que l'application n'est ps statique et que le changement est constant dans le programme.
\subsection{Travail en équipe}
Le travail en Open-Space est une expérience très enrichissante, permettant de communiquer et de partager plus facilement avec les membres d'une équipe. Même si il a aussi ses inconvénients. Cela m'a permis d'avancer plus facilement que si j'avais été dans un bureau isolé.
\subsection{Méthodes Agiles}
L'initiation à la méthode Agile en particulier à l'eXtreme Programming m'a fait découvrir d'autres façons de programmer en équipe. L'utilisation régulière des tests unitaires et tests d'acceptation m'ont fait comprendre qu'il était possible de connaitre à tout moment l'état de santé du code source d'une application. Permettant ainsi de le remanier ou de l'éditer de plusieurs endroits à la fois sans l'endommager d'un autre endroit.

\section{Observations}
Ce stage m'a permis d'apprécier le fonctionnement d'une équipe de petite taille. Cela implique que certains des développeurs jouaient plusieurs rôles (developpement, admin, maintenance de l'intégration continue, gestion du site web, du serveur de mails). Je suis curieux de voir comment cela est géré dans une entreprise de plus grande taille, eventuellement au cours de mon stage de fin d'atudes.
\subparagraph*{}
Les différentes façons de programmer et de gérer le cycle de vie du logiciel que j'ai pu rencontrer pendant ce stage sont efficaces à leur manière. Cependant je me suis rendu compte qu'elles ne sont pas toute adaptable dans tous les cas ni dans toutes les équipes. Il est important de garder en mémoire qu'en chaque contexte le méthodes à utiliser varient.

\section{Apport à l'entreprise d'accueil}
Je ne pense pas pouvoir quantifier financièrement le fruit de mon travail au sein de l'équipe.