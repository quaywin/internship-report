\documentclass[12pt,oneside]{book}
\usepackage{makeidx,anysize,mflogo,xspace,texnames,float,epsfig}
\usepackage{amsmath}
\usepackage{amsfonts}
\usepackage{amssymb}
\usepackage[utf8]{inputenc}
\usepackage[TS1,OT1,T1]{fontenc}
\usepackage[francais]{babel} %\usepackage[english]{babel}
\usepackage{graphicx} % Pour les figures
\usepackage{colordvi} % Pour la couleur
\usepackage{listings} % Pour les listes
\usepackage[usenames,dvipsnames]{color}
\usepackage{hyperref} % Pour les liens hypertexts
%\usepackage{algorithm} %\usepackage{algorithmic} % Pour des algorithmes
%\usepackage{algpseudocode}
\usepackage[font=small,format=plain,labelfont=bf,up,textfont=it,up]{caption} % Config des légendes (figure, tableau, ..)

\usepackage{fancyhdr} % Pour l'entête et les pieds de pages

%------------------------------------------------------------------------
% Configuration des entêtes et pieds de pages
\pagestyle{fancy}
\renewcommand{\chaptermark}[1]{\markboth{#1}{}}
\renewcommand{\sectionmark}[1]{\markright{\thesection\ #1}}
\fancyhf{} \fancyhead[LE,RO]{\bfseries\thepage}
\fancyhead[LO]{\bfseries\rightmark}
\fancyhead[RE]{\bfseries\leftmark}
\renewcommand{\headrulewidth}{0.5pt}
\addtolength{\headheight}{0.5pt}
\renewcommand{\footrulewidth}{0pt}

%------------------------------------------------------------------------
% Indentation et coloration syntaxique du language XML
\lstset{language=XML,
basicstyle=\small,
keywordstyle=\bf \color{blue},
identifierstyle=\underline,
commentstyle=\color[gray]{0.5},
stringstyle=\color{blue},
showstringspaces=false}

%------------------------------------------------------------------------

\newcommand{\clearemptydoublepage}{% Nouvelle commande pour laisser une page blanche
	\newpage{\pagestyle{empty}\cleardoublepage}}

%------------------------------------------------------------------------
% Commandes pour la page de garde
\makeatletter
\def\clap#1{\hbox to 0pt{\hss #1\hss}}%

\def\ligne#1{
	\hbox to \hsize{\vbox{\centering #1}}}%
	
\def\haut#1#2#3{%
	\hbox to \hsize{%
	\rlap{\vtop{\raggedright #1}}%
	\hss
	\clap{\vtop{\centering #2}}%
	\hss
	\llap{\vtop{\raggedleft #3}}}}%
	
\def\bas#1#2#3{%
	\hbox to \hsize{%
	\rlap{\vbox{\raggedright #1}}%
	\hss
	\clap{\vbox{\centering #2}}%
	\hss
	\llap{\vbox{\raggedleft #3}}}}%
	
\def\maketitle{%
	\thispagestyle{empty}\vbox to \vsize{%
		\haut{\@logoUniversity}{}{\@logoLabo}
		\vspace{.7cm}
		\haut{}{\@blurb}{}
		\vfill
		\vspace{.5cm}
		\begin{flushleft}
		\usefont{OT1}{ptm}{m}{n}
		\huge \@title
		\end{flushleft}
		\par
		\hrule height 4pt
		\par
		\begin{flushright}
		\usefont{OT1}{phv}{m}{n}
		\Large \@author
		\par
		\end{flushright}
		\vspace{1cm}
		\haut{}{\@logoCentral}{}
		\vfill
		\vfill
		\bas{}{\@location, le \@date}{}}%
	\cleardoublepage
}

\def\date#1{\def\@date{#1}}
\def\author#1{\def\@author{#1}}
\def\title#1{\def\@title{#1}}
\def\location#1{\def\@location{#1}}
\def\blurb#1{\def\@blurb{#1}}
\def\logoCentral#1{\def\@logoCentral{#1}}
\def\logoUniversity#1{\def\@logoUniversity{#1}}
\def\logoLabo#1{\def\@logoLabo{#1}}
%------------------------------------------------------------------------
% Initialisation par défaut
\date{\today}
\author{}
\title{}
\location{}
\blurb{}
\logoCentral{}
\logoUniversity{}
\logoLabo{}
\makeatother

%------------------------------------------------------------------------
 
%\setlength{\textwidth}{160mm} % LARGEUR
%\setlength{\parindent}{10mm} % Taille des retraits

%------------------------------------------------------------------------
% Configuration de la page de garde
\title{Introduction d'un modeleur open source dans la solution Smartesting}
\author{Cyril \textsc{Moutenet}}
\date{\today}
\blurb{UFR Sciences et techniques de Franche-Comté\\ \ \\
\textbf{Rapport de stage de\\2ème année de Master Informatique\\
Option Systèmes distribués et réseaux}\\[1em]
Maître de stage : Olivier Albiez\\
Tuteur universitaire : Fabien Peureux
}
\logoCentral{%
	\includegraphics[height=3cm]{images/logo-smartesting.png}}
\logoUniversity{%
	\includegraphics[height=2.5cm]{images/logo-ufc.jpg}}
\logoLabo{%
	\includegraphics[height=2.5cm]{images/logo-lifc.png}}
\location{Besançon}

%------------------------------------------------------------------------

\begin{document}
	\begin{frontmatter}
    	\maketitle
		%Page blanche
		\clearemptydoublepage
		\chapter*{}
		% Remerciements
		\chapter*{Remerciements}
\addcontentsline{toc}{chapter}{Remerciements} 

Je tiens à remercier tout d'abord mon maître de de stage Olivier Albiez pour sa disponibilité, les conseils et toutes les choses qu'il m'a appris.

\subparagraph*{}
Un grand merci pour toute l'équipe de recherche et développement de Smartesting avec qui j'ai pu connaitre une expérience unique en son genre et des plus instructives.


		\tableofcontents 
	\end{frontmatter}

	%Contenu du rapport
	\chapter{Introduction}

Il s'agit dans ce chapitre, de présenter la société Smartesting, son produit développé et, le contexte de travail dans lequel le stage de Master Professionnel Informatique option Systèmes Distribués et Réseaux de l'université de franche-comté a eu lieu.

\section{Présentation de l'entreprise}

Le stage s’est déroulé dans la société Smartesting située à Temis (figure \ref{Temis}) de Besançon. Il s'agit d'une société née il y a six ans à partir de travaux de recherches menés au LIFC\footnote{LIFC : Laboratoire d'Informatique de Franche-Comté} sur la génération de tests. 
Ce sont Laurent Py et Bruno Legeard qui, en 2003, ont ainsi créé la société Leirios qui deviendra  Smartesting en juin 2008.
Celle-ci emploie plus d'une trentaine de personnes dont dix en R\&D\footnote{R\&D : Recherche et développement} où mon stage a eu lieu.

\begin{figure}[!ht]
\begin{center}
  \includegraphics[height=5cm]{images/temis.jpg}
  \caption{Temis innovation}
  \label{Temis}
\end{center}
\end{figure}

Smartesting est un éditeur de logiciel de génération automatique des cas de test à partir d'une modélisation UML des exigences. Smartesting utilise des outils de modélisation du marché basés sur Eclipse. Deux outils de modélisation sont supportés : IBM Rational Software Modeler\footnote{RSM ou équivalent comme IBM Rational Software Architect} et Borland Together. L'interface avec Smartesting se fait par l'intermédiaire d'un <<plug-in>> Eclipse spécifique à Smartesting (cf. figure \ref{SolutionSmartesting} p. \pageref{SolutionSmartesting}).

\begin{figure}[!ht]
\begin{center}
  \includegraphics[width=.7\textwidth]{images/TestDesigner.png}
  \caption{Solution Smartesting}
  \label{SolutionSmartesting}
\end{center}
\end{figure}

\subparagraph*{}
La société est organisée en trois services distincts qui sont :
\begin{itemize}
  \item les commerciaux
  \item les consultants
  \item l'équipe R\&D
\end{itemize}
 Durant le stage, j'ai rencontré l'ensemble des employés de Smartesting mais j'ai principalement travaillé avec la R\&D.

\subparagraph*{}
La R\&D où s'est déroulé mon stage, était un environnement de développement ``Agile''. 
Il s'agit d'une méthode basée sur le développement itératif, où les exigences et les solutions évoluent grâce à la collaboration. 
La gestion du projet avec les méthodes Agiles encourage les processus d'inspections et de remise en question de l'équipe de façon fréquente. 
En générale, la philosophie des responsables est l'encouragement de l'équipe, l'auto-organisation et les responsabilités. 
Ensemble qui pousse l'utilisation de bonnes pratiques de programmation et la mise en place de systèmes de livraison rapide, fiable et en adéquation avec les besoins clients et l'objectif de la société.
Ceci étant obtenu grâce à un réseau de communication rapproché et un processus de fabrication sans gaspillage.

\section{La solution Smartesting}

De manière plus précise (cf. \ref{figure:TheSolutionSmartesting} p.\pageref{figure:TheSolutionSmartesting}), à partir des besoins métier et des exigences de test, est réalisé un modèle UML.
Ce modèle peut être uniquement réalisé à partir des deux modeleurs RSM et Together.
Il s'agit de modeleurs basés sur la plateforme Eclipse. Eclipse est une architecture de plugins qui communiquent les uns avec les autres pour former une application. C'est via ce mécanisme que Smartesting a développé deux plugins d'exportation de modèle pour les deux modeleurs précédemment cités.

\subparagraph*{}
Le logiciel ``Test Designer'' va quant à lui, générer des tests de manière automatique à partir du modèle exporté.
Le résultat des tests est stocké dans un référentiel de test de Test Designer.
Puis, ces tests vont être publiés dans un environnement de test tiers.

\begin{figure}[!ht]
\begin{center}
  \includegraphics[width=\textwidth]{images/TheSolutionSmartesting.png}
  \caption{La solution Smartesting complète}
  \label{figure:TheSolutionSmartesting}
\end{center}
\end{figure}

\subparagraph*{}
Smartesting peut publier vers plusieurs environnements de gestion de tests. Ces environnements permettent par exemple de suivre l'évolution du test tout au long de son cycle de vie. Ensuite grâce à une couche d'adaptation, l'application modélisée pourra être testée.

\subparagraph*{}
Le processus de génération de tests de la solution Smartesting est un processus itératif. C'est à dire que si l'application déjà testée doit évoluée, la prise en compte des nouvelles fonctionnalités, ne générera pas un nouveau coût de génération des tests. 
Il suffit de modifier le modèle UML de spécifications via le modeleur, puis de générer les tests à nouveau (automatique).

\subparagraph*{}
Traditionnellement la génération de tests est effectuée manuellement par un ingénieur de tests.

\section{Objectifs}

L’objectif de ce stage est d'intégrer un nouveau modeleur, mais cette fois open source, à l'ensemble des modeleurs compatibles avec la solution Smartesting.
Et c'est à la suite du projet VETESS\footnote{VETESS : Vérification de systèmes embarqués VEhicules par génération automatique de TESts à partir des Spécifications}, dans lequel Smartesting est partenaire, que le choix du modeleur Papyrus a eu lieu.

\subparagraph*{\includegraphics[height=1cm]{images/logo-vetess.png} :}
L’objectif stratégique du projet VETESS est de produire des outils conceptuels, méthodologiques et techniques pour la vérification de systèmes mécatroniques embarqués véhicule par génération automatique de tests à partir des spécifications de ces systèmes.


\subparagraph*{}
Pour remplir ces objectifs, le projet VETESS s’appuie sur un partenariat fortement complémentaire avec une expertise importante dans l’ingénierie dirigée par les modèles et la génération automatique de tests (figure \ref{figure:PartenairesVetess} p.\pageref{figure:PartenairesVetess}). 
Il réunit un industriel soucieux de maîtriser la complexité des systèmes grand public (PSA Peugeot Citroën), à une PME innovante (Smartesting) leader dans le domaine du test à partir de modèle et, à un industriel Clemessy spécialisé possédant une offre de premier plan en matière de bancs de test de système électriques et électroniques embarqués. 
Au niveau académique, les laboratoires impliqués (Université de Haute Alsace – Laboratoire MIPS et Université de Franche-Comté – Equipe LIFC) sont reconnus respectivement dans le domaine de la modélisation logicielle et système, et de la génération de tests.

\begin{figure}[!ht]
\begin{center}
  \includegraphics[width=\textwidth]{images/PartenaireVetess.png}
  \caption{Partenaires VETESS}
  \label{figure:PartenairesVetess}
\end{center}
\end{figure}

\subparagraph{La solution Smartesting par rapport à VETESS :}
Il est important que Smartesting puisse proposer sa solution aux entreprises qui souhaitent valider des systèmes embarqués.
C'est pour cela qu'intégrer le modeleur choisit dans le projet VETESS est un atout commercial important pour Smartesting.
Il s'agit d'offrir la possibilité a ses clients, d'utiliser le modeleur Papyrus comme modeleur de la solution.
Sur la figure \ref{ObjectifPapyrus} p.\pageref{ObjectifPapyrus} l'objectif en image, propose trois plugins d'exportation de modèle.

\begin{figure}[!ht]
\begin{center}
  \includegraphics[height=5cm]{images/objectif_papyrus2.png}
  \caption{Objectif : intégration du modeleur Papyrus}
  \label{ObjectifPapyrus}
\end{center}
\end{figure}

\subparagraph*{}
A la différence de la figure \ref{ObjectifPapyrus} p.\pageref{ObjectifPapyrus}, il ne s'agit pas de recréer obligatoirement un plugin spécifique à Papyrus. 
Il faut étudier le modeleur Papyrus afin de constater des éventuels points communs avec les modeleurs existants.
C'est à partir de l'étude des points communs avec les autres modeleurs qu'il a été évoqué le besoin de fragmenter l'existant en plugins.

\subparagraph{Fragmentation en plugins :}
La fragmentation en plugins est un objectif secondaire. 
Cela doit permettre la réorganisation des modules en plugins, afin de pouvoir utiliser du code commun.
La ``pluginisation'' de l'application a posé quelques problèmes techniques.
En particulier, la construction du produit appelé le \build a nécessité un gros chantier d'amélioration.

\subparagraph*{}
Le \build est un terme utilisé pour définir le processus automatisé qui permet de construire et déployer la solution Smartesting sur différentes plateformes telles que Linux et Windows, et sur différents modeleurs comme \together, \rsm et ``Papyrus''.
Pour Smartesting, ce sont des tâches \textit{ant} qui réalisent l'automatisation du processus de construction du produit.

L'amélioration du \build et la modification de l'infrastructure des modules doivent permettre d'atteindre le genre d'architecture de la figure \ref{reorganisation} (p.\pageref{reorganisation}).

\begin{figure}[!h]
\begin{center}
  \includegraphics[width=\textwidth]{images/reorganisation2.png}
  \caption{Réorganisation en plugins}
  \label{reorganisation}
\end{center}
\end{figure}

L'amélioration du \build doit permettre de créer les plugins mais aussi de les déployer. La figure \ref{reorganisation} montre les différents modules à déployer avec le \build :
\begin{itemize}
  \item Le plugin (en bleu)
  \item La feature (en violet)
  \item L'update site (en orange)
\end{itemize}
\ \\
Ces trois types de modules permettent de déployer un plugin sur un modeleur type, grâce à un update-site.

\subparagraph*{}
La figure \ref{reorganisation} montre le travail complet de réorganisation de l'architecture de l'application. 
En outre, l'utilisation des points d'extension a permis de faciliter l'intégration du spécifique dans les modeleurs.
Grâce à ce mécanisme, il est possible d'adapter le comportement d'un plugin sans ajouter de dépendances entre le plugin standard et le plugin spécifique. 
Exemple sur la figure \ref{reorganisation} p.\pageref{reorganisation} le plugin ``Exporter for RSM'' utilise le point d'extension du plugin ``Common extractor'' (point noir).
En faisant cela, le plugin ``Common extractor'' (point rouge) est capable d'interroger le plugin ``Exporter for RSM'' afin qu'il réalise une extraction spécifique.
Cela permet d'éviter de créer une dépendance au plugin spécifique \rsm. Ceci est aussi appelé une inversion de dépendance.

\section{Méthode de travail}

Le déroulement du stage s'est effectué au sein d'une équipe de développement ``Agile'' ou eXtrem Programming (XP).
Cette méthode valorise la cohésion d'équipe et l'auto-organisation. 
La fabrication de quelque chose doit toujours être réalisée sans contrainte, dans des conditions dites de ``fun'', terme extrait d'un livre dont le titre est <<Art of Agile development>>. 
Il s'agit d'une lecture que j'ai faite durant le stage pour préparer les discussions collectives sur quelques chapitres. 

\subparagraph*{}
Dès le départ, ce qu’il faut comprendre de la méthode ``Agile'' est que le processus de développement peut être sans cesse amélioré. 
Il y a quelques fois un groupe de lecture et chaque semaine une rétrospective qui donne lieu à des discussions. 
Celles-ci permettent d’améliorer le processus de développement.
C’est ainsi que j'ai pu participer à certains changements.

La semaine est organisée en plusieurs événements répartis sur cette dernière et des réunions chaque midi appelées également morning meeting ou standup meeting.

\subsection{Morning meeting}

Chaque midi, l'équipe R\&D se réunit avant la pause pour faire le point sur le travail en cours de chacun. Tout le monde peut participer à cette réunion, pas seulement les membres de l'équipe R\&D. Cette réunion très courte permet de prendre connaissance du travail de ses collègues. Elle permet également de passer une annonce.

\subparagraph*{}
Ensuite, la semaine débute par une rétrospective, genre de bilan d’une semaine de travail, qui représente aussi en méthode XP une itération. 

\subsection{Rétrospective}

La rétrospective est une réunion qui dure environ une heure et qui permet de s’exprimer et d’évacuer les frustrations éventuelles ressenties lors de l’itération passée.
Chaque semaine un animateur a pour rôle de faire participer tous les employés R\&D et de time boxer\footnote{Time boxé : Expression pour dire par exemple que l’on contrôle le temps de la réunion} la réunion. Les stagiaires ni sont pas exempts. 

\subparagraph*{}
Enfin, le travail de la semaine est planifié par le client XP à la fin de la rétrospective. Éventuellement, celui-ci peut demander une séance de cotation pour estimer le coût de développement d’une ou plusieurs fonctionnalités appelées aussi fiches blanches.

\subsection{Séance de cotation}

Le travail est organisé à partir de fiches représentant le travail à réaliser. Il peut s’agir d’une fonctionnalité à développer, d'un bug à corriger ou d'un travail d’exploration à effectuer.

\subparagraph*{}
Une fois les fiches cotées, le client XP peut proposer le travail à faire pour l’itération lors de l’engagement.

\subsection{L'engagement}

Le client XP dispose sur un tableau les fiches à réaliser par l’équipe de développement en fonction de l’effort que peut produire celle-ci. 
Cet effort est un nombre de points définis par la R\&D. 
Il s’agit de la vélocité, le client XP ne peut dépasser cette valeur.

\subparagraph*{}
Chez Smartesting, un point correspond au travail effectué durant une demi-journée sur une fiche sans interruption.

\subsection{Pair Programming}

Le ``Pair programming'' est une pratique qui consiste à réaliser une tâche par binôme. 
Ce procédé permet une meilleure qualité du travail accompli, de prendre de meilleures décisions, d'aller plus vite que tout seul et surtout il permet d'éviter de rendre quelqu'un indispensable.

\subparagraph*{}
Le travail en paire a déjà été pratiqué à l'Université lors de projets.
Mais cela prend une autre dimension en entreprise lorsque l'on travaille avec quelqu'un qui connaît bien son sujet et qui nous l'explique. 
Très vite, on connaît le principe de fonctionnement de l'application et l'on peut commencer à participer aux cotations de fiches.

\subsection{Batman et Robin}

Le batman est le nom donné à la personne qui joue le rôle de superviseur de l'itération.
Il est là pour surveiller le déroulement des fiches.
Il discute avec les développeurs afin de vérifier ce qui va être réalisé ou ce qui est en cours de réalisation.

\subparagraph*{}
Il peut poser des questions au client XP à la place des développeurs.
Il peut aussi demander au client XP d'aller voir les développeurs pour ré-expliquer ce qu'il attend.
Avec le client XP, il peut préparer la démonstration pour la rétrospective de l’itération.

\subparagraph*{}
Il est associé au veilleur (Robin) qui aide certain binôme dans la réalisation de leur fiche.
Le veilleur peut être amené à faire rencontrer des développeurs, car il a jugé que cela pourrait les aider à mieux avancer dans leur travail.

\subparagraph*{}
Au delà du côté humain de la méthode de travail ``Agile'', il y a toute l'infrastructure de développement du logiciel. Celle-ci permet de garantir la fiabilité et la non régression du logiciel développé.

\subsection{Intégrateur continu}

L'intégration continue est un élément essentiel gage de qualité fonctionnelle du programme développé.
L'intégrateur est chargé d'exécuter en continu plusieurs tâches :
\begin{itemize}
  \item Les tests unitaires
  \item Les tests fonctionnels (Test fit)
  \item Les tests haut niveau (High level)
  \item Les tests smokes
\end{itemize}
\ \\
La durée de détection d'un bug varie suivant la tâche exécutée. Les tests sont répartis sur plusieurs ordinateurs pour être exécutés en parallèle.

\subparagraph*{}
Les tests unitaires sont les plus rapides à détecter les erreurs de bas niveau.
Pour arriver jusqu'aux tests dit smoke qui permettent de vérifier que l'application déployée fonctionne correctement. 
Exemple, ce test peut détecter un problème de déploiement de librairie ou l'utilisation d'une librairie non compatible avec tel ou tel modeleur. 

\subparagraph*{}
Pour avoir une plus grande réactivité, un panneau lumineux représentant les grandes catégories de test définies plus haut, a été installé pour visualiser plus rapidement un test en échec.
Ensuite, c'est à chacun de s'autogèrer pour réparer au plus tôt le problème.

\section{Synthèse}

L'objectif du stage dans la société Smartesting est principalement l'intégration du modeleur open source papyrus dans la solution Smartesting.
Il y a cependant un autre objectif secondaire, qui est la fragmentation de l'existant en plugins.
Ces deux objectifs sont étroitement liés, puisque sans la fragmentation en plugin, il n'est pas possible de créer un nouveau plugin pour Papyrus sans duplication de code.
On remarque aussi que la fragmentation en plugin a généré des problèmes sur la construction du produit Smartesting.

\subparagraph*{}
En fait, l'objectif principal et secondaire de ce stage, sont aussi des objectifs pour Smartesting.
Grâce à la fragmentation par exemple, Smartesting se rapproche d'avantage d'une intégration complète dans Eclipse et donne la possibilité de développer d'autres plugins plus aisément.

\subparagraph*{}
Voyons maintenant dans le second chapitre, comment le modeleur Papyrus fonctionne et quelles améliorations ont permis la fragmentation en plugins.


\chapter{Methodes Agiles}\label{agile}
L'équipe dans laquelle je fais mon stage utilisent les méthodes Agiles(cf. lexique \ref{lexique:agile} p.\pageref{lexique:agile}) et eXtreme Programming(cf. lexique \ref{lexique:XP} p.\pageref{lexique:XP}) en particulier. Ces méthodes ne sont pas figées et doivent être adaptées à chaque équipe. Il est donc courant et primordial que certaines pratiques soient remises en cause (cf. tableau \ref{tableau:evolPratXP} p.\pageref{tableau:evolPratXP}. Ce chapitre donne un aperçu des pratiques XP utilisées par l'équipe R\&D de SMARTESTING dans le cadre du développement de la solution Smartesting.

% Le projet est stocké sur un serveur Subversion\footnote{}
%TRAC, SVN, HUDSON, Jira, IntelliJ Idea, Ubuntu 8.10, RSM 7.0.0.x, 7.0.5, 7.5, Together 2007,2008, RQM, HP Mercury Quality Center, HP Quick Test Center.
%Travail sur IDEA, svn
%TODO : mise en place, utilisation, utilité ...

\section{Itération}
L'équipe organise son travail sous forme d'itérations d'une semaine. Chaque itération traite un nombre de fonctionnalités limité qui est quantifié en points de vélocité(cf. lexique \ref{lexique:velocité} p.\pageref{lexique:velocité}). À la fin de mon stage la vélocité variait entre 7 et 13 points par semaine. Une itération est planifiée lors de la rétrospective de l'itération précédente. Un itération compte un certain nombre de points de vélocité qui constitue une estimation de la durée requise pour développer les fonctionalités planifiées. La durée d'une itération à la première semaine de mon stage etait de deux semaines. Elle est tout de suite passée à une semaine. À la fin d'une itération le produit est livré. Les versions mineures sont disponibles pour les consultants afin de les tester sur le terrain et d'obtenir des retours en condition réele. Les versions majeures de fin de jalon (Milestone) bénéficient d'une période de validation approfondie et sont disponibles aux utilisateurs finaux.

\begin{figure}[!h]
\centering
\includegraphics[scale=0.10]{Illustrations/SP_A0182.jpg}
\caption{Iteration}
\label{fig:Iteration}
\end{figure}

\section{Rétrospective}
Lors de la rétrospective d'une itération, les membres de l'équipe de R\& D se réunissent pour parler de la précédente itération et pour planifier celle à venir. La réunion commence par un ``check in'' pendant lequel une question est posée (par exemple ``Comment voyez vous Test Designer dans un an?'') en général afin de se remémorer l'itération (se remettre mentalement dedans). Un point est sur des statistiques (métriques) telles que le nombre de lignes de code,la couverture de tests unitaires(cf. lexique \ref{lexique:testU} p.\pageref{lexique:testU}) et haut niveau(cf. lexique \ref{lexique:testHL} p.\pageref{lexique:testHL}) ou encore la vélocité atteinte par rapport aux objectifs. L'attention est ensuite portée sur les fonctionnalités qui ont été réalisées ou non lors de l'itération. La réunion se continue par les discussions diverses, chacun peut discuter de diverses choses : apparition de problèmes, amélioration du fonctionnement de l'équipe. La réunion se termine enfin par l'annonce de la prochaine vélocité et l'attribution des responsabilités pour l'itération qui vient. En effet chaque itération a des responsables différents (animateur de la rétrospective, validation, veilleur ...)

\section{Fiches}\label{agile:fiches}
Des fiches correspondent aux différentes tâches à effectuer dans le cadre d'une itération. Elles peuvent être de type différent : valeur client (couleur blanche), Tâche technique (couleur verte), Point technique (couleur bleue), Anomalie (couleur rouge), Amélioration de process / Prototype (couleur jaune). Les couleurs des fiches permettent à l'équipe d'identifier au premier coup d'oeil le travail à réaliser et l'urgence. Si le tableau comporte beaucoup de fiches rouges, elles seront à traiter en priorité car ce sont des anomalies. Les fiches possèdent des points de vélocité qui correspondent à la quantité de travail à effectuer (en demie-journée par binôme) sur la tâche. Si la fiche est trop importante elle peut être redécoupées en fiches plus petites.

\begin{figure}[!h]
\centering
\includegraphics[scale=0.10]{Illustrations/SP_A0185.jpg}
\caption{Fiches}
\label{fig:Fiches}
\end{figure}

\begin{figure}[!h]
\centering
\includegraphics[scale=0.10]{Illustrations/SP_A0183.jpg}
\caption{Tableau d'avancement des fiches}
\label{fig:Tableau d'avancement des fiches}
\end{figure}

\section{Travail en binômes (pair programming)}
Une grand partie du travail dans l'équipe est réalisée en binôme. Les binômes changent très souvent. Les avantages du travail en binôme sont multiples. Contrairement à la programmation individuelle les binômes permettent de diffuser plus rapidement le savoir acquis lors du développement et seront plus à même à partager leur connaissances (effet boule de neige). De plus des études montrent que le travail par paires donne généralement un code de meilleure qualité, une meilleure capture des bugs. La communication et la motivation dans l'équipe sont également améliorées par cette méthode. Chaque personne dans le binôme peut prendre le clavier et la souris pour donner ses idées de développement. En général le travail en binôme est un échange d'idées permanent. Chaque binôme, en début de fiche s'engage sur cette fiche et estime le temps qui lui sera necessaire pour la terminer. Cette estimation est revue régulièrement (lors du morning meeting par exemple) et est communiquée à l'équipe et au client XP.

\begin{figure}[!h]
\centering
\includegraphics[scale=0.15]{Illustrations/SP_A0188.jpg}
\caption{Pair programming}
\label{fig:Pair programming}
\end{figure}

\section{Validation}
La validation joue un rôle important lors de l'itération. Une partie de la validation est réalisée en permanence par les serveurs d'intégration continue. Une autre partie (en général, l'utilisation via l'interface homme-machine) ne peut être réalisée que par des personnes qui n'ont pas participé au développement de la fonctionnalité. Cela permet d'avoir une idée plus objective des manipulations à effectuer pour tester la fonctionnalité. Au cours de la semaine la validation est la tâche exclusive de Batman. Il valide une fiche dès qu'elle a été mise ``À valider'' afin d'avoir le retour le plus rapide sur la fonctionnalité. Lors de la validation, on fait très souvent appel au client XP pour vérifier que la fonctionnalité correspond bien aux besoin énoncés. À la fin de l'itération, lorsque toute les fiche sont terminées et validées, l'équipe commence une validation de l'ensemble des fiches de l'itération. Lorsque toute l'équipe est satisfaite (les bugs éventuels corrigés, documentation relue ...) la livraison peut avoir lieu.

\section{Semaine type}
La semaine Agile à la R\&d de SMARTESTING est rythmée par quelques pratiques qui peuvent évoluer.
\subsection*{Cotation et engagement}
Avant de commencer une itération, l'egagement est fait. C'est à dire que les fiches qui vont déterminer les développements de la semaine vont être choisies par le client XP. Les fiches sont posées sur le tableau en vue de tous. La somme des vélocités des fiches correspond à la vélocité déterminée lors de la retrospective de l'itération précédente. L'équipe peut alors choisir les fiches qui seront commencées dans la semaine, les binômes se créent. Chaque binôme va ensuite voir le client XP afin de confirmer le périmètre fonctionnel de la fiche. 

\subsection{Morning meeting}
Les membres de l'équipe se réunissent juste avant la pause déjeuner pour parler de ce qu'ils ont fait la matinée. Toute l'équipe est réunie en cercle et se fait passer un objet. Celui qui a l'objet prend la parole. Chaque binôme informe toute l'équipe de l'avancée de son engagement ainsi que des problèmes qu'il rencontre. Le temps de parole est très bref et si un problème mérite une attention particulière il sera traité hors du cercle plus tard. Après le tour du cercle s'il y a des informations complémentaires à mentionner elles le sont. Le meeting se termine ensuite par une pensée du jour.

\subsubsection{Point perso}
Le mardi, juste avant la pause déjeuner, un membre de l'équipe fait une présentation à l'aide d'un projecteur sur un sujet de son choix (pas nécessairement lié à l'informatique d'ailleurs). La présentation doit durer 10 minutes. À la fin, les membres de l'équipe peuvent poser des questions et commenter la manière dont la présentation a été réalisée (Intéresser le public, parler clairement, qualité du support...). Cet exercice de communication permet aux membres de l'équipe d'apprendre à partager et à structurer leurs idées. Cela peut être très utile lors de manifestations diverses (xp days, agile tour...). Le point perso a évolué entre le début et la fin de mon stage, et actuellement différentes nouvelles formes en sont expérimentées.

\subsubsection{Discussion sur un livre}
L'équipe lit un livre pendant la semaine et se réunit le mercredi avant la pause de midi pour commenter un chapitre. L'équipe lorsque je suis arrivé lisait ``The Art of Agile Development''. Cette activité permet deja de lire en anglais , mais également d'améliorer la cohésion de l'équipe par l'éventuelle adoption de pratiques nouvelles. Plusieurs des pratiques évoquées dans ce livre ont été adoptées après mon arrivée. Ce fut le cas de ``No bugs'', ``Done-done'', ``Slack time'' par exemple. Cette pratique a disparu avec le temps car plusieurs personnes de l'équipe n'y trouvaient plus d'intérêt.

\subsection{Point technique}
Le Jeudi matin en début de matinée a lieu le point technique il peut varier dans son contenu. Cela peut aller de la discussion d'un point technique rencontré lors de l'itération à un code contest de 30 minutes.



\chapter{Déroulement de mon travail}
Pendant mon stage j'ai participé à beaucoup de fiches différentes, il serait long et ennuyeux de les détailler toutes sachant que je participai à entre 1 et 4 fiches par semaine. Je présenterai dans les grandes lignes les fonctionnalités majeures auxquelles j'ai participé. 
\section{Développements sur le code de production}
Parmi les activités qui m'ont été confiées, j'ai participé à la réalisation de fiches blanches (cf. \ref{agile:fiches} p.\pageref{agile:fiches}). Ce code est appelé ``code de production'' car il couvre les besoins fonctionnels exprimés par le client XP. La première étape dans le travail sur une fiche est de préciser le périmètre fonctionnel avec le client XP. Ceci permet aux développeurs de connaitre son besoin précis. Ensuite, dans la mesure du possible, on crée les tests unitaires qui permettront de savoir si la fonctionnalité est opérationnelle. Les membres du binôme peuvent prendre le clavier au moment où ils le souhaitent et proposer leurs idées. Chaque direction prise et ainsi réfléchie et validée par les deux du binôme.
\subsection{Observations}
\subparagraph*{}
Au début de mon stage, la première fonctionnalité sur laquelle j'ai travaillé fut des observations. Dans le produit, le observations sont des stéréotypes\footnote{stéréotypes au sens UML} attachés à des opérations de classes du modèle de test. Les observations sont déclenchées par des triggers\footnote{Événement déclencheur} selon une pré-condition décrite dans le langage OCL\footnote{OCL (Object Constraint Language) est un langage informatique d'expression des contraintes utilisé par UML.}. Dans les modeleurs, à l'export, ce stéréotype est reconnu et intégré au modèle propre à Test Designer. Les observations font office de ``points de contrôle '' permettant de vérifier que le système sous test réagit correctement.
\subparagraph*{}
Les observations sont introduites dans les deux modeleurs principaux de façon différente. Dans RSM, les stéréotypes et la gestion des triggers spécifiques à Smartesting proviennent d'un profil personnalisé. L'utilisation des profils dans le modeleur Together ont été l'occasion d'une autre fiche à laquelle j'ai contribué. Cependant mon binôme et moi-même nous somme rendu compte, après avoir passé 4 points de vélocité sur cette fiche cotée à 2, que c'était plus compliqué que nous l'imaginions. Nous avons alors suspendu la fiche en attente de conseils d'OBEO\footnote{OBEO est une société de conseil experte dans le domaine de la modélisation EMF/GMF sous Eclipse}. Finalement la après leur réponse, nous avons dû ``Rollback'' \footnote{revenir à la version précédente} notre travail. La solution adaptée finalement fut d'utiliser le code préexistant. C'est à dire utiliser des propriétés ``Custom''(cf. figure \ref{figure:obsTriggerTG} p.\pageref{figure:obsTriggerTG}) pour gérer les triggers dans Together. Ces triggers servent à déterminer les opérations que l'on souhaite observer. Une fois définies dans le modèle UML et après export, les observations sont visibles dans Test Designer (cf. figure\ref{figure:obsTD} p.\pageref{figure:obsTD}).
\begin{figure}[!ht]
\centering
\includegraphics[scale=0.5]{Illustrations/Observation_Trigger_Together.png}
\caption{Intégration des observations dans Together}
\label{figure:obsTriggerTG}
\end{figure}
\begin{figure}[!ht]
\centering
\includegraphics[scale=0.5]{Illustrations/Observation.png}
\caption{Intégration des observations dans Test Designer}
\label{figure:obsTD}
\end{figure}
\subsection{Descriptions}
\subparagraph*{}
Dans test Designer, plusieurs éléments ont besoin d'avoir une description attachée afin d'être exploitables à la publication, par exemple dans la publication HTML, HP Mercury Quality Center ou encore dans Rational Quality Manager. A l'origine les descriptions qui étaient déjà implémentées étaient publiées en texte brut. Après un court spike (cf. lexique \ref{lexique:spike} p.\pageref{lexique:spike}) nous nous sommes aperçu que plusieurs publishers supportaient des balises HTML. Il était également possible de mettre du texte en forme en amont dans les modeleurs (cf. tableau \ref{tableau:compatDescHTML} p. \pageref{tableau:compatDescHTML}). Néanmoins certaines versions de modeleurs n'ont pas les mêmes standards de mise en forme des descriptions. Ainsi il a fallu passer par une étape intermédiaire pour unifier les descriptions dans un format spécifique interne. Il est par exemple nécessaire d'enlever certaines balises en trop ou encore de convertir des balises en d'autres plus globalement supportées. Nous avons choisi d'utiliser le format XHTML avec l'aide de la bibliothèque JTidy (cf \ref{figure:descXHTMLPublisher} p. \pageref{figure:descXHTMLPublisher}). Les descriptions obtenues sont ensuite ``rendues'' pour être affichées ainsi qu'on le souhaite dans les différents publishers.
\begin{table}[!ht]
\caption{\label{tableau:compatDescHTML}Compatibilité HTML des modeleurs et publishers}
\begin{tabular}{|l|l|l|}
\hline
Type & Nom & Balises supportées (simplifié) \\
\hline
\hline
\multirow{4}{*}{Modeleurs} & Together 2007 & HTML <b><i><u> \\
& RSM 7.0.0.x & texte brut\\
& RSM 7.0.5.x & HTML <b><i><u> + centrer + paragraphes\\ 
& RSM 7.5 & HTML <b><i><u>+ centrer + paragraphes + autres\\ \hline
\multirow{3}{*}{Publishers} & HTML & \\
& Quality Center & texte brut et HTML <b><i><u> \\
& Specification (pdf) & <strong><em><ins> \\
& Prototype RQM & HTML <b><i><u> \\ \hline
\end{tabular}
\end{table}
\begin{figure}[!ht]
\centering
\includegraphics[scale=0.5]{Illustrations/bigDescSchema.png}
\caption{Passage des descriptions des modeleurs à leur publication}
\label{figure:descXHTMLPublisher}
\end{figure}
\subsection{Test Suites}
Les suites de test permettent d'avoir une unité structurante pour tout un ensemble de test à partir des modeleurs. Elles définissent le périmètre des tests. Auparavant, une suite de test était considérée implicitement dans le modèle. Les différents packages\footnote{Équivalent d'un dossier pour stocker des ressources dans Eclipse dans le cadre de la modélisation} contenant des informations relatives aux suites étaient analysés puis elles étaient personnalisées (filtre) dans le Target Manager. 
\subparagraph*{}
En fin du dernier jalon, guidé par des besoins de pouvoir remonter des informations des suites jusque là difficilement accessibles. Le client XP après plusieurs discussions avec l'équipe a décidé qu'il fallait externaliser les suites de test dans des fichiers de type texte. Cette modification de la structure des suite doit permettre d'être une première étape pour unrendre la solution Smartesting ``verticale'' afin de pouvoir par exemple s'intégrer à SAP. La solution choisie a été d'utiliser un fichier texte écrit dans un langage facile à écrire et à comprendre : Yaml\footnote{``YAML Ain't Markup Language''; il s'agit d'un langage dédié à la serialisation (cf. lexique \ref{lexique:serialisation} p.\pageref{lexique:serialisation})}. Yaml est dit ``user friendly'' facile à lire et à comprendre pour un utilisateur.
\begin{figure}[!ht]
\centering
\ttfamily{
\small{
\begin{verbatim}
# un commentaire et un tableau
users:
- Toto
- Titi
# utilisation de booléens
vrai: true
faux: false
# tableaux associatifs multidimensionnels
foo:
toto: gentil
titi: mechant
\end{verbatim}
}
}
\caption{Exemple de fichier YAML}
\label{figure:exYaml}
\end{figure}
\begin{figure}[!ht]
\centering
\ttfamily{
\small{
\begin{verbatim}
--- !TestSuite
identifier: 4d517188-80af-4255-9405-abc367dcb5a7
initialModelInstance: initial_model_instance
version: "1.0"
\end{verbatim}
}
}
\caption{Exemple de fichier YAML pour les TestSuite}
\label{figure:exTestSuite}
\end{figure} 
\subparagraph*{}
Malgré sa simplicité apparente, l'extraction des suite de test a été difficile. Tout d'abord l'équipe ayant peu de connaissances dans Eclipse sur certains domaines relatifs à cette fonctionnalité, il a fallu un peu de temps d'adaptation. Les différences entre les modeleurs ont rendu la tâche encore plus difficile. Plusieurs étapes ont été nécessaires lors de l'élaboration de cette fonctionnalité. Un fois le standard de format de fichier établi pour le suites de test, il a été possible de créer des outils permettant de prendre en compte ce fichier lors de l'extraction du modèle. Une interface graphique de type ''wizard`` a été créée pour faciliter la saisie des données par l'utilisateur.
\subparagraph*{}
D'un autre coté un autre binôme travaillait sur gestion de l'écoute des fichiers de suites de test. Ceci a pour but de s'assurer que les données inscrites dans l'éditeur du fichier de suites de test sont bien mises à jours lors de la modification des noms de package, leur déplacement ou autre. Cette fonctionalité étant difficile à mettre en place sans générer des familles de bugs gênantes, elle a été abandonée.
\subparagraph*{}
Les suites de test ont fait l'objet d'une release 3.4.1 peu de temps après la release majeure du jalon de la version 3.4. Cette livraison a eu lieu en début de mois de juillet.
\section{Correction de bugs}
La correction de bugs fait partie du travail des développeurs même si ce n'est pas forcément agréable. Toutefois les tests, la validation, l'intégration continue et les itérations courtes rendent la détection de bugs très rapide.La detection des bugs au plus tôt est primordiale poue éviter les coûts supplémentaires (patch, mises à jour, perte d'utilisateurs...) après les livraions. La plupart des bugs sont en général détectés et corrigés avant la livraison en fin de semaine. Dans l'équipe on considère qu'il existe deux types de bugs. Les bugs client sont des bugs qui ont été remontés par les utilisateurs finaux après une release officielle. Les autres bugs qui sont détectés pendant l'itération ou après des releases mineures ne sont pas considérés comme des bugs client. Cette distinction est importante car, en adoptant la pratique ``No Bugs'' l'équipe s'engage à livrer un produit de qualité avec un minimum d'anomalies. L'équipe R\&D s'est fixée comme objectif pour l'année d'avoir moins de 40 bugs client. Cet engagement influe sur la prime de l'équipe.
\section{Documentation}
Certaines fonctionnalités nécessitent d'être documentées dans le guide de l'utilisateur. La documentation de Test Designer est entièrement en anglais au format HTML. Les parties relatives à l'outil lui même et aux plugins dans les modeleurs peuvent être accompagnées de captures d'écran lorsque cela est nécessaire. Dans ce cas il arrive très souvent que la documentation soit dédoublée dans le cas des différents modeleurs. Mis à part le guide de l'utilisateur, il est nécessaire de mettre à jour les release notes lors de la livraison d'un version destinée à des utilisateurs finaux. La documentation, une fois réalisée est soumise à une relecture par une autre personne.
\section{Validation}
La validation joue un rôle important lors de l'itération. Une partie de la validation est réalisée en permanence par les serveurs d'intégration continue. Une autre partie (en général, l'utilisation via l'interface homme-machine) ne peut être réalisée que par des personnes qui n'ont pas participé au développement de la fonctionnalité. Cela permet d'avoir une idée plus objective des manipulations à effectuer pour tester la fonctionnalité. Au cours de la semaine la validation est la tâche exclusive de Batman. Il valide une fiche dès qu'elle a été mise ``À valider'' afin d'avoir le retour le plus rapide sur la fonctionnalité. Lors de la validation, on fait très souvent appel au client XP pour vérifier que la fonctionnalité correspond bien aux besoin énoncés. À la fin de l'itération, lorsque toutes les fiche sont terminées et validées, l'équipe commence une validation de l'ensemble des fiches de l'itération. Lorsque toute l'équipe est satisfaite (les bugs éventuels corrigés, documentation relue ...) la livraison peut avoir lieu.
\section{Amélioration du code existant (refactoring)}
Le refactoring est une pratique courante dans les développements de l'équipe R\&D de Smartesting. Il consiste à rendre le code plus facile à maintenir ou à lire sans pour autant modifier les fonctionnalités. Cela peut se faire de différentes manières. Le code peut être modifié en vue d'être auto-commenté (cf. \ref{autoComm} p.\pageref{autoComm}). Il peut être découpé en d'autres méthodes et classes pour être plus facile à tester unitairement. Il peut aussi être déplacé dans d'autres packages.
\subparagraph*{}
Le refactoring a lieu dans le cadre de travail sur du code de production lors de fiches blanches. Il peut également être mis en pratique lors de réduction de dette de code\footnote{La dette de code est un phénomène qui a lieu lorsqu'un code source produit devient de plus en plus dur à maintenir du fait de modifications successives ou d'un mauvais design. Ce phénomène est à éviter à tout prix car plus la dette de code est grande plus il est difficile de la réduire.} pendant le ''slack-time``.
\section{Prototypes}
L'équipe s'est engagée à livrer un certain nombre de prototypes par an. Sous l'appelation de prototype, on sous entend des fonctionnalités qui ont une bonne valeur pour les utilisateurs mais qui ne sont pas prévue dans la roadmap. Les développeurs profitent donc du slack-time pour développer leurs prototypes. Cela peut aller de l'ajout de raccourci dans des menus contextuels pour rendre l'utilisation plus simple à un algorithme de génération de test pouvant couvrir toutes les exigences d'un modèle de test. J'ai participé au developpement d'un prototype permettant d'afficher de façon graphique les tests. Ce prototype permet d'avoir un aperçu rapide des test pour pouvoir les optimiser.
\section{Administration système}
Sur une courte période j'ai effectué des t\^aches d'administration système. En particulier au moment de l'intégration des Google Apps dans le fonctionnement de Smartesting. La necessité de partager des calendriers et de pouvoir y accéder via des plateformes mobiles a amené Smartesting à envisager d'utiliser Google Apps. Ainsi, en binôme avec Olivier, nous avons appréhendé le panneau d'administration ainsi que les différents services utilisables. Chaque calendrier donne la possibilité d'être exporté, ainsi nous avons pu réaliser une routine de backup\footnote{sauvegarde automatique}.
\section{Réunions}
\subsection{Meeting corporate}
Le meeting corporate a lieu tous les quatre mois chez Smartesting. Cette série de journées est dédiée au bilan du quart précédent et aux engagements sur le prochain quart. Les engagements sont divers et vont des ventes (objectifs en chiffre d'affaires) aux développeurs (fonctionnalités). J'ai assisté à deux meetings corporate au cours de mon stage. Le premier meeting corporate a eu lieu du 8 au 10 avril, le second du 6 au 8 juillet. Les différents collaborateurs de Smartesting sont réunis lors de ces journées. La première journée commence par une réunion générale résumant dans les grandes ligne le bilan du quart précédent et certains objectifs du suivant. Les commerciaux, les avant-vente, les consultants, les développeurs et le directoire peuvent ainsi avoir des nouvelles sur chaque service de l'entreprise. Les objectifs de ventes sont exposés, les résultats de la fin du Jalon sont commentés, les opportunités de contrats avec de nouveaux clients sont présentées.
\subsection{Amélioration du process, évolution du fonctionnement de l'équipe}
J'ai été amené à l'occasion de rétrospectives ou de réunions à réfléchir sur le processus de développement au sein de l'équipe. L'équipe de R\&D est très concernée par l'amélioration du processus et toute pratique peut être remise en cause si elle ne convient pas a l'équipe. Chaque décision, quelle qu'elle soit, doit être approuvée par toute l'équipe avant d'être prise. L'évolution des pratiques a lieu en permanence et entre le début et la fin de mon stage beaucoup de pratiques sont apparues et d'autres ont disparu (cf. tableau \ref{tableau:evolPratXP} p.\pageref{tableau:evolPratXP}).
\subparagraph*{}
L'ensemble des réunions ont amené l'équipe à proposer et remettre en question certaines pratiques lorsque le besoin s'en ressentait. La séance de lecture hebdomadaire du livre ``The Art of Agile Developpment'' a ammené plusieurs pratiques telles que ``Done-done'', ``No-bugs'', ``Slack-Time''. Des discussions lors de rétrospectives ont entrainé l'expérimentation pendant un jour de la technique du pomodoro (cf. annexe \ref{annexe:pomodoro} p.\pageref{annexe:pomodoro}). L'expérimentation a continué pendant une semaine, puis a été adoptée jusqu'à nouvel ordre.
Tableau des évolution des pratiques XP
\begin{table}[!ht]
	\caption{\label{tableau:evolPratXP}Evolution des pratiques XP au cours du stage}
	\begin{tabular}{|l|c|c|}
		\hline
		Pratique & Début du stage & Fin du stage\\
		\hline
		Pair programming & \tick & \tick \\
		Itération & 2 semaines & 1 semaine \\
		Intégration continue & \tick & \tick \\
		Niko Niko & \tick & \badtick \\
		Lecture & \tick & \badtick \\
		Point perso & hebdomadaire & nouvelles experimentations\\
		Pomodoro & \badtick & \tick \\
		Test driven development & \tick & \tick \\
		``Done done'' & théorique & adopté et normalisé\\
		``No Bugs'' & imprécis & engagement \\
		Slack Time & \badtick & adopté et normalisé\\
		Rétrospective & 1 à 2h le lundi matin & Timeboxée et juste après la livraison\\
		Point technique & assez réguliers & moins nombreux\\
		Veilleur & \tick & Robin cumule son rôle\\
		Batman/Robin & \badtick & \tick \\
		\hline
	\end{tabular}
\end{table}

\subparagraph*{}
Plus recemment on recherche à devenir une ``équipe performante'' au sens de Lean(cf. lexique \ref{lexique:lean} p.\pageref{lexique:lean}). Certaines valeurs de Lean sont intéressante à utiliser pour renforcer l'équipe :
\begin{itemize}
\item prendre soin de soi
\item être connecté avec les autres
\item avoir une vision partagée
\item avoir conscience que le produit est à l'image de l'équipe, inversement l'équipe est à l'image du produit
\end{itemize}
\subparagraph*{}
Certains standards de l'équipe R\&D ont été rédigés et affichés. Même s'ils sont souvent voués à changer, cela permet de se rappeler rapidement les pratiques qui ont été adoptées par l'équipe. Si une pratique n'est plus utilisée par toute l'équipe cela signifie qu'elle est peut être vouée à disparaître.
\subsection{Évaluation}
Lors de mon stage j'ai accepté de me faire évaluer comme si j'étais un salarié de Smartesting. L'évaluation est bi-annuelle, elle a pour but de faire connaitre au directoire comment se sent chaque membre de l'équipe. Un suivi est établi d'évaluation en évaluation.

\chapter{Conclusion}
\subsection{Connaissances acquises}
Programmation JAVA
Utilisation de tests unitaires\ref{lexique:TestU}.
Design patterns
Travail en Open-Space
Initiation à la méthode Agile en particulier à l'eXtreme Programming

\subsection{Apport à l'entreprise d'accueil}
\section{ce que j'ai apporté}
\section{ce que j'ai appris}
\section{difficultés}
blabla

	
	%ANNEXE
	\appendix % Numérotation de l'annexe en A,B,C,...
		
	%\addcontentsline{toc}{chaper}{Annexes}
	%\chapter*{Annexes} 
	
	%Résumé
	\pagestyle{empty} %Pas de numéro de page

\section*{Résumé}
Ce rapport résume les quatre mois de stage passés dans la société Smartesting, au sein d'une équipe R\&D ``Agile''.
Smartesting est un éditeur de génération automatique de tests à partir d'une modélisation UML des exigences. C'est à partir de modeleurs non open source basé sur une plateforme d'Eclipse que la modélisation est effectuée jusqu'alors.
 
\subparagraph*{}
Le but de ce stage fut l'introduction du modeleur open source Papyrus dans la solution Smartesting.
En créant un nouveau plugin pour celui-ci, Smartesting a souhaité modifier le processus de construction de l'application afin de permettre de s'implanter d'avantage sur la plateforme Eclipse.
\subparagraph*{}
En remplissant tous les objectifs fixés du stage cela m'a permis de découvrir, le framework de modélisation Emf, la plateforme Eclipse, les mécanismes d'extension de plugin et la méthode ``Agile''.

\section*{Mots-clés}
Méthode ``Agile'', Open source, Eclipse, modeleur, Rsm, Together, Plugin, feature, update-site, OSGI, EMF
\vfill

\section*{Abstract}

This report summarizes about four months of training spent in the company Smartesting, with an ``Agile'' R\&D team. Smartesting is a software editor which automates test case generation from the functional model of a UML specification, with closed source modelers based on the Eclipse platform.
\subparagraph*{}
The goal of the training period was the introduction of an open source modeler in the Smartesting solution. With the new plugin creation, the company wanted to modify the build process to take advantage of the Eclipse platform.
\subparagraph*{}
Achieving these objectives allowed we to discover the Emf framework to study the Eclipse platform, and to take part in ``Agile'' development.

\section*{Key words}
Agile method, Open source, Eclipse, modeler, Rsm, Together, Plugin, feature, update-site, OSGI, EMF
\end{document}
